\let\mypdfximage\pdfximage
\def\pdfximage{\immediate\mypdfximage}

\documentclass[12pt]{article}
\usepackage[table]{xcolor}
\usepackage[margin=1in]{geometry} 
\usepackage{amsmath,amsthm,amssymb}
\usepackage[english]{babel}
\usepackage{graphicx}
\usepackage{tcolorbox}
\usepackage{enumitem}
\usepackage{hyperref}
\usepackage{fmtcount}
\usepackage{listings}
\usepackage{blkarray}
\usepackage{float}
\usepackage{bm}
\usepackage{subfigure}
\usepackage{booktabs}
\usepackage[maxfloats=256]{morefloats}
\usepackage{siunitx}
\usepackage{forloop}

\usepackage{geometry}
\geometry{a4paper,
          left=            5mm,
          right=           5mm,
          top=             15mm,
          bottom=          25mm,
          heightrounded}

\maxdeadcycles=10000

\setcounter{secnumdepth}{5}
\setcounter{tocdepth}{5}

\newtheorem{theorem}{Theorem}[section]
\newtheorem{corollary}{Corollary}[theorem]
\newtheorem{lemma}[theorem]{Lemma}
\newtheorem{proposition}[theorem]{proposition}
\newtheorem{exmp}{Example}[section]\newtheorem{definition}{Definition}[section]
\newtheorem{remark}{Remark}
\newtheorem{ex}{Exercise}
\theoremstyle{definition}
\theoremstyle{remark}
\bibliographystyle{elsarticle-num}

\DeclareMathOperator{\sinc}{sinc}
\newcommand{\RNum}[1]{\uppercase\expandafter{\romannumeral #1\relax}}
\newcommand{\N}{\mathbb{N}}
\newcommand{\Z}{\mathbb{Z}}
\newcommand{\R}{\mathbb{R}}
\newcommand{\E}{\mathbb{E}}
\newcommand{\matindex}[1]{\mbox{\scriptsize#1}}
\newcommand{\V}{\mathbb{V}}
\newcommand{\Q}{\mathbb{Q}}
\newcommand{\K}{\mathbb{K}}
\newcommand{\C}{\mathbb{C}}
\newcommand{\prob}{\mathbb{P}}
\newcommand{\Date}{This text will change :D}
\newcommand{\DateDay}{This text will change :(}
\newcommand{\DateMonth}{This text will change :)}

\lstset{numbers=left, numberstyle=\tiny, stepnumber=1, numbersep=5pt}

\begin{document}
\title{Uruguayan Historicals of Production\\
and Simulations}
%\author{Renzo Miguel Caballero Rosas} 
\maketitle

\newpage

\newcounter{i}
\newcounter{j}
\newcounter{k}

\forloop{i}{2018}{\value{i} < 2020}{
\forloop{j}{1}{\value{j} < 13}{%13
\forloop{k}{1}{\value{k} < 32}{%32

% I have somehow 4 cases. I have to consider that the date 20190101 must be written as i0j0k. But the date 20191010 should be ijk. For this reason, I added the 4 cases ijk, i0jk, ij0k, i0j0k.

% To test the ifnum:
%\ifnum\i<2020
%\arabic{i}
%\else
%FALSE
%\fi

\IfFileExists{../../wsJobs/ws_allData_20191022/Simulations/Simulation_\arabic{i}\arabic{j}\arabic{k}/120.eps}{
\graphicspath{{../../wsJobs/ws_allData_20191022/Simulations/Simulation_\arabic{i}\arabic{j}\arabic{k}/}}
%\newpage
\begin{table}[ht!]
\centering
\begin{tabular}{cccc}
\toprule
\multicolumn{4}{c} \textbf{\arabic{i}-\arabic{j}-\arabic{k}:} \\
\midrule
\multicolumn{2}{c}{\includegraphics[scale=0.2]{101.eps}} & \multicolumn{2}{c}	{\includegraphics[scale=0.2]{120.eps}}\\
\includegraphics[scale=0.25]{102.eps} & 	\includegraphics[scale=0.25]{103.eps} & 	\includegraphics[scale=0.25]{104.eps} & 	 \includegraphics[scale=0.25]{110.eps} \\
%\includegraphics[scale=0.35]{109.eps} & 	\includegraphics[scale=0.35]{110.eps} & 	\includegraphics[scale=0.35]{111.eps} & 	\includegraphics[scale=0.35]{112.eps} \\
\includegraphics[scale=0.25]{107.eps} &
\includegraphics[scale=0.25]{106.eps} & 	\includegraphics[scale=0.25]{108.eps} & 	\includegraphics[scale=0.25]{127.eps} \\
\includegraphics[scale=0.25]{113.eps} & 	\includegraphics[scale=0.25]{121.eps} & 	\includegraphics[scale=0.25]{112.eps} & 	\includegraphics[scale=0.25]{119.eps} \\
\multicolumn{2}{c}{\includegraphics[scale=0.3]	{116.eps}} & \multicolumn{2}{c}	{\includegraphics[scale=0.3]{117.eps}}\\
\multicolumn{2}{c}{\includegraphics[scale=0.3]	{124.eps}} & \multicolumn{2}{c}	{\includegraphics[scale=0.3]{123.eps}}\\
\bottomrule
\end{tabular}
\end{table}
}{}

% I added the \ifnum because: The loop considered two times the date 2019-01-05. The correct one, and the extra one when i-0j-k, e.g., i=2019, j=10, and k=5.
\ifnum \value{k}>9
\IfFileExists{../../wsJobs/ws_allData_20191022/Simulations/Simulation_\arabic{i}0\arabic{j}\arabic{k}/120.eps}{
\graphicspath{{../../wsJobs/ws_allData_20191022/Simulations/Simulation_\arabic{i}0\arabic{j}\arabic{k}/}}
%\newpage
\begin{table}[ht!]
\centering
\begin{tabular}{cccc}
\toprule
\multicolumn{4}{c} \textbf{\arabic{i}-\arabic{j}-\arabic{k}:} \\
\midrule
\multicolumn{2}{c}{\includegraphics[scale=0.2]{101.eps}} & \multicolumn{2}{c}	{\includegraphics[scale=0.2]{120.eps}}\\
\includegraphics[scale=0.25]{102.eps} & 	\includegraphics[scale=0.25]{103.eps} & 	\includegraphics[scale=0.25]{104.eps} & 	 \includegraphics[scale=0.25]{110.eps} \\
%\includegraphics[scale=0.35]{109.eps} & 	\includegraphics[scale=0.35]{110.eps} & 	\includegraphics[scale=0.35]{111.eps} & 	\includegraphics[scale=0.35]{112.eps} \\
\includegraphics[scale=0.25]{107.eps} &
\includegraphics[scale=0.25]{106.eps} & 	\includegraphics[scale=0.25]{108.eps} & 	\includegraphics[scale=0.25]{127.eps} \\
\includegraphics[scale=0.25]{113.eps} & 	\includegraphics[scale=0.25]{121.eps} & 	\includegraphics[scale=0.25]{112.eps} & 	\includegraphics[scale=0.25]{119.eps} \\
\multicolumn{2}{c}{\includegraphics[scale=0.3]	{116.eps}} & \multicolumn{2}{c}	{\includegraphics[scale=0.3]{117.eps}}\\
\multicolumn{2}{c}{\includegraphics[scale=0.3]	{124.eps}} & \multicolumn{2}{c}	{\includegraphics[scale=0.3]{123.eps}}\\
\bottomrule
\end{tabular}
\end{table}
}{}
\fi

\IfFileExists{../../wsJobs/ws_allData_20191022/Simulations/Simulation_\arabic{i}\arabic{j}0\arabic{k}/120.eps}{
\graphicspath{{../../wsJobs/ws_allData_20191022/Simulations/Simulation_\arabic{i}\arabic{j}0\arabic{k}/}}
%\newpage
\begin{table}[ht!]
\centering
\begin{tabular}{cccc}
\toprule
\multicolumn{4}{c} \textbf{\arabic{i}-\arabic{j}-\arabic{k}:} \\
\midrule
\multicolumn{2}{c}{\includegraphics[scale=0.2]{101.eps}} & \multicolumn{2}{c}	{\includegraphics[scale=0.2]{120.eps}}\\
\includegraphics[scale=0.25]{102.eps} & 	\includegraphics[scale=0.25]{103.eps} & 	\includegraphics[scale=0.25]{104.eps} & 	 \includegraphics[scale=0.25]{110.eps} \\
%\includegraphics[scale=0.35]{109.eps} & 	\includegraphics[scale=0.35]{110.eps} & 	\includegraphics[scale=0.35]{111.eps} & 	\includegraphics[scale=0.35]{112.eps} \\
\includegraphics[scale=0.25]{107.eps} &
\includegraphics[scale=0.25]{106.eps} & 	\includegraphics[scale=0.25]{108.eps} & 	\includegraphics[scale=0.25]{127.eps} \\
\includegraphics[scale=0.25]{113.eps} & 	\includegraphics[scale=0.25]{121.eps} & 	\includegraphics[scale=0.25]{112.eps} & 	\includegraphics[scale=0.25]{119.eps} \\
\multicolumn{2}{c}{\includegraphics[scale=0.3]	{116.eps}} & \multicolumn{2}{c}	{\includegraphics[scale=0.3]{117.eps}}\\
\multicolumn{2}{c}{\includegraphics[scale=0.3]	{124.eps}} & \multicolumn{2}{c}	{\includegraphics[scale=0.3]{123.eps}}\\
\bottomrule
\end{tabular}
\end{table}
}{}

\IfFileExists{../../wsJobs/ws_allData_20191022/Simulations/Simulation_\arabic{i}0\arabic{j}0\arabic{k}/120.eps}{
\graphicspath{{../../wsJobs/ws_allData_20191022/Simulations/Simulation_\arabic{i}0\arabic{j}0\arabic{k}/}}
%\newpage
\begin{table}[ht!]
\centering
\begin{tabular}{cccc}
\toprule
\multicolumn{4}{c} \textbf{\arabic{i}-\arabic{j}-\arabic{k}:} \\
\midrule
\multicolumn{2}{c}{\includegraphics[scale=0.2]{101.eps}} & \multicolumn{2}{c}	{\includegraphics[scale=0.2]{120.eps}}\\
\includegraphics[scale=0.25]{102.eps} & 	\includegraphics[scale=0.25]{103.eps} & 	\includegraphics[scale=0.25]{104.eps} & 	 \includegraphics[scale=0.25]{110.eps} \\
%\includegraphics[scale=0.35]{109.eps} & 	\includegraphics[scale=0.35]{110.eps} & 	\includegraphics[scale=0.35]{111.eps} & 	\includegraphics[scale=0.35]{112.eps} \\
\includegraphics[scale=0.25]{107.eps} &
\includegraphics[scale=0.25]{106.eps} & 	\includegraphics[scale=0.25]{108.eps} & 	\includegraphics[scale=0.25]{127.eps} \\
\includegraphics[scale=0.25]{113.eps} & 	\includegraphics[scale=0.25]{121.eps} & 	\includegraphics[scale=0.25]{112.eps} & 	\includegraphics[scale=0.25]{119.eps} \\
\multicolumn{2}{c}{\includegraphics[scale=0.3]	{116.eps}} & \multicolumn{2}{c}	{\includegraphics[scale=0.3]{117.eps}}\\
\multicolumn{2}{c}{\includegraphics[scale=0.3]	{124.eps}} & \multicolumn{2}{c}	{\includegraphics[scale=0.3]{123.eps}}\\
\bottomrule
\end{tabular}
\end{table}
}{}

}
}
}

\end{document}