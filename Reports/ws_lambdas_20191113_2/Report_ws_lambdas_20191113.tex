\let\mypdfximage\pdfximage
\def\pdfximage{\immediate\mypdfximage}

\documentclass[12pt]{article}
\usepackage[table]{xcolor}
\usepackage[margin=1in]{geometry} 
\usepackage{amsmath,amsthm,amssymb}
\usepackage[english]{babel}
\usepackage{graphicx}
\usepackage{tcolorbox}
\usepackage{enumitem}
\usepackage{hyperref}
\usepackage{fmtcount}
\usepackage{listings}
\usepackage{blkarray}
\usepackage{float}
\usepackage{bm}
\usepackage{subfigure}
\usepackage{booktabs}
\usepackage[maxfloats=256]{morefloats}
\usepackage{siunitx}
\usepackage{forloop}
%\usepackage[space]{grffile}

% To do computations with the counters:
\usepackage{calc}

\usepackage{geometry}
\geometry{a4paper,
          left=            5mm,
          right=           5mm,
          top=             15mm,
          bottom=          25mm,
          heightrounded}

\maxdeadcycles=10000

\setcounter{secnumdepth}{5}
\setcounter{tocdepth}{5}

\newtheorem{theorem}{Theorem}[section]
\newtheorem{corollary}{Corollary}[theorem]
\newtheorem{lemma}[theorem]{Lemma}
\newtheorem{proposition}[theorem]{proposition}
\newtheorem{exmp}{Example}[section]\newtheorem{definition}{Definition}[section]
\newtheorem{remark}{Remark}
\newtheorem{ex}{Exercise}
\theoremstyle{definition}
\theoremstyle{remark}
\bibliographystyle{elsarticle-num}

\DeclareMathOperator{\sinc}{sinc}
\newcommand{\RNum}[1]{\uppercase\expandafter{\romannumeral #1\relax}}
\newcommand{\N}{\mathbb{N}}
\newcommand{\Z}{\mathbb{Z}}
\newcommand{\R}{\mathbb{R}}
\newcommand{\E}{\mathbb{E}}
\newcommand{\matindex}[1]{\mbox{\scriptsize#1}}
\newcommand{\V}{\mathbb{V}}
\newcommand{\Q}{\mathbb{Q}}
\newcommand{\K}{\mathbb{K}}
\newcommand{\C}{\mathbb{C}}
\newcommand{\prob}{\mathbb{P}}
\newcommand{\Date}{This text will change :D}
\newcommand{\DateDay}{This text will change :(}
\newcommand{\DateMonth}{This text will change :)}
\newcommand{\mysize}{0.28}
\newcommand{\mysizeTwo}{0.20}


\lstset{numbers=left, numberstyle=\tiny, stepnumber=1, numbersep=5pt}

\begin{document}
\title{Uruguayan Historicals of Production\\
and Simulations}
%\author{Renzo Miguel Caballero Rosas} 
\maketitle

\graphicspath{{../../wsJobs/ws_lambdas_20191113/TEST_Matlab-Python/Optimization_OC_Subgradient_Method/Simulations_Output_Modified_03122019/}}

\begin{figure}[H]
\centering
\includegraphics[scale=0.8]{DG_20180302.eps}
\end{figure}

\newcounter{i}
\newcounter{j}

\forloop{i}{1}{\value{i} < 8}{

%\newpage
\begin{table}[ht!]

\centering

\setcounter{j}{7*(\value{i}-1)+1}
\IfFileExists{../../wsJobs/ws_lambdas_20191113/TEST_Matlab-Python/Optimization_OC_Subgradient_Method/Simulations_Output_Modified_03122019/Simulation_20180302_\arabic{j}/120.eps}{
\graphicspath{{../../wsJobs/ws_lambdas_20191113/TEST_Matlab-Python/Optimization_OC_Subgradient_Method/Simulations_Output_Modified_03122019/Simulation_20180302_\arabic{j}/}}
\begin{tabular}{ccccc}
\toprule
\includegraphics[scale=\mysize]{1000.eps} & \includegraphics[scale=\mysize]{1001.eps} & \includegraphics[scale=\mysize]{1002.eps} & \includegraphics[scale=\mysize]{1003.eps} & \includegraphics[scale=\mysize]{1004.eps}\\
\bottomrule
\end{tabular}
}{}

\setcounter{j}{7*(\value{i}-1)+2}
\IfFileExists{../../wsJobs/ws_lambdas_20191113/TEST_Matlab-Python/Optimization_OC_Subgradient_Method/Simulations_Output_Modified_03122019/Simulation_20180302_\arabic{j}/120.eps}{
\graphicspath{{../../wsJobs/ws_lambdas_20191113/TEST_Matlab-Python/Optimization_OC_Subgradient_Method/Simulations_Output_Modified_03122019/Simulation_20180302_\arabic{j}/}}
\begin{tabular}{ccccc}
\toprule
\includegraphics[scale=\mysize]{1000.eps} & \includegraphics[scale=\mysize]{1001.eps} & \includegraphics[scale=\mysize]{1002.eps} & \includegraphics[scale=\mysize]{1003.eps} & \includegraphics[scale=\mysize]{1004.eps}\\
\bottomrule
\end{tabular}
}{}

\setcounter{j}{7*(\value{i}-1)+3}
\IfFileExists{../../wsJobs/ws_lambdas_20191113/TEST_Matlab-Python/Optimization_OC_Subgradient_Method/Simulations_Output_Modified_03122019/Simulation_20180302_\arabic{j}/120.eps}{
\graphicspath{{../../wsJobs/ws_lambdas_20191113/TEST_Matlab-Python/Optimization_OC_Subgradient_Method/Simulations_Output_Modified_03122019/Simulation_20180302_\arabic{j}/}}
\begin{tabular}{ccccc}
\toprule
\includegraphics[scale=\mysize]{1000.eps} & \includegraphics[scale=\mysize]{1001.eps} & \includegraphics[scale=\mysize]{1002.eps} & \includegraphics[scale=\mysize]{1003.eps} & \includegraphics[scale=\mysize]{1004.eps}\\
\bottomrule
\end{tabular}
}{}

\setcounter{j}{7*(\value{i}-1)+4}
\IfFileExists{../../wsJobs/ws_lambdas_20191113/TEST_Matlab-Python/Optimization_OC_Subgradient_Method/Simulations_Output_Modified_03122019/Simulation_20180302_\arabic{j}/120.eps}{
\graphicspath{{../../wsJobs/ws_lambdas_20191113/TEST_Matlab-Python/Optimization_OC_Subgradient_Method/Simulations_Output_Modified_03122019/Simulation_20180302_\arabic{j}/}}
\begin{tabular}{ccccc}
\toprule
\includegraphics[scale=\mysize]{1000.eps} & \includegraphics[scale=\mysize]{1001.eps} & \includegraphics[scale=\mysize]{1002.eps} & \includegraphics[scale=\mysize]{1003.eps} & \includegraphics[scale=\mysize]{1004.eps}\\
\bottomrule
\end{tabular}
}{}

\setcounter{j}{7*(\value{i}-1)+5}
\IfFileExists{../../wsJobs/ws_lambdas_20191113/TEST_Matlab-Python/Optimization_OC_Subgradient_Method/Simulations_Output_Modified_03122019/Simulation_20180302_\arabic{j}/120.eps}{
\graphicspath{{../../wsJobs/ws_lambdas_20191113/TEST_Matlab-Python/Optimization_OC_Subgradient_Method/Simulations_Output_Modified_03122019/Simulation_20180302_\arabic{j}/}}
\begin{tabular}{ccccc}
\toprule
\includegraphics[scale=\mysize]{1000.eps} & \includegraphics[scale=\mysize]{1001.eps} & \includegraphics[scale=\mysize]{1002.eps} & \includegraphics[scale=\mysize]{1003.eps} & \includegraphics[scale=\mysize]{1004.eps}\\
\bottomrule
\end{tabular}
}{}

\setcounter{j}{7*(\value{i}-1)+6}
\IfFileExists{../../wsJobs/ws_lambdas_20191113/TEST_Matlab-Python/Optimization_OC_Subgradient_Method/Simulations_Output_Modified_03122019/Simulation_20180302_\arabic{j}/120.eps}{
\graphicspath{{../../wsJobs/ws_lambdas_20191113/TEST_Matlab-Python/Optimization_OC_Subgradient_Method/Simulations_Output_Modified_03122019/Simulation_20180302_\arabic{j}/}}
\begin{tabular}{ccccc}
\toprule
\includegraphics[scale=\mysize]{1000.eps} & \includegraphics[scale=\mysize]{1001.eps} & \includegraphics[scale=\mysize]{1002.eps} & \includegraphics[scale=\mysize]{1003.eps} & \includegraphics[scale=\mysize]{1004.eps}\\
\bottomrule
\end{tabular}
}{}

\setcounter{j}{7*(\value{i}-1)+7}
\IfFileExists{../../wsJobs/ws_lambdas_20191113/TEST_Matlab-Python/Optimization_OC_Subgradient_Method/Simulations_Output_Modified_03122019/Simulation_20180302_\arabic{j}/120.eps}{
\graphicspath{{../../wsJobs/ws_lambdas_20191113/TEST_Matlab-Python/Optimization_OC_Subgradient_Method/Simulations_Output_Modified_03122019/Simulation_20180302_\arabic{j}/}}
\begin{tabular}{ccccc}
\toprule
\includegraphics[scale=\mysize]{1000.eps} & \includegraphics[scale=\mysize]{1001.eps} & \includegraphics[scale=\mysize]{1002.eps} & \includegraphics[scale=\mysize]{1003.eps} & \includegraphics[scale=\mysize]{1004.eps}\\
\bottomrule
\end{tabular}
}{}

\end{table}

}

\end{document}