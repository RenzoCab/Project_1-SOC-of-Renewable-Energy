%%%%%%%%%%%%%%%%%%%%%%%%%%%%%%%%%%%%%%%%%
% Beamer Presentation
% LaTeX Template
% Version 1.0 (10/11/12)
%
% This template has been downloaded from:
% http://www.LaTeXTemplates.com
%
% License:
% CC BY-NC-SA 3.0 (http://creativecommons.org/licenses/by-nc-sa/3.0/)
%
%%%%%%%%%%%%%%%%%%%%%%%%%%%%%%%%%%%%%%%%%

%------------------------------------------------
%	PACKAGES AND THEMES
%------------------------------------------------

\documentclass[aspectratio=169]{beamer}

\mode<presentation> {

% The Beamer class comes with a number of default slide themes
% which change the colors and layouts of slides. Below this is a list
% of all the themes, uncomment each in turn to see what they look like.

%\usetheme{default}
%\usetheme{AnnArbor}
%\usetheme{Antibes}
%\usetheme{Bergen}
%\usetheme{Berkeley}
%\usetheme{Berlin}
%\usetheme{Boadilla}
%\usetheme{CambridgeUS}
%\usetheme{Copenhagen}
%\usetheme{Darmstadt}
%\usetheme{Dresden}
%\usetheme{Frankfurt}
%\usetheme{Goettingen}
%\usetheme{Hannover}
%\usetheme{Ilmenau}
%\usetheme{JuanLesPins}
%\usetheme{Luebeck}
\usetheme{Madrid}
%\usetheme{Malmoe}
%\usetheme{Marburg}
%\usetheme{Montpellier}
%\usetheme{PaloAlto}
%\usetheme{Pittsburgh}
%\usetheme{Rochester}
%\usetheme{Singapore}
%\usetheme{Szeged}
%\usetheme{Warsaw}

% As well as themes, the Beamer class has a number of color themes
% for any slide theme. Uncomment each of these in turn to see how it
% changes the colors of your current slide theme.

%\usecolortheme{albatross}
%\usecolortheme{beaver}
%\usecolortheme{beetle}
%\usecolortheme{crane}
%\usecolortheme{dolphin}
%\usecolortheme{dove}
%\usecolortheme{fly}
%\usecolortheme{lily}
%\usecolortheme{orchid}
%\usecolortheme{rose}
%\usecolortheme{seagull}
%\usecolortheme{seahorse}
%\usecolortheme{whale}
%\usecolortheme{wolverine}

%\setbeamertemplate{footline} % To remove the footer line in all slides uncomment this line
%\setbeamertemplate{footline}[page number] % To replace the footer line in all slides with a simple slide count uncomment this line

\setbeamertemplate{navigation symbols}{} % To remove the navigation symbols from the bottom of all slides uncomment this line
}

\usepackage{graphicx} % Allows including images
\usepackage{booktabs} % Allows the use of \toprule, \midrule and \bottomrule in tables
\usepackage{subfig}
\usepackage{bm}
\usepackage{array}
\usepackage{commath}
\usepackage{makecell}
\usepackage{xcolor}
\usepackage{graphicx}
\usepackage{amsmath}
\usepackage{wasysym}
\usepackage{mathtools}
\usepackage[linesnumbered,ruled,vlined]{algorithm2e}
\newcommand{\E}{\mathbb{E}}
\newcommand{\R}{\mathbb{R}}
\usepackage{siunitx}

%------------------------------------------------
%	TITLE PAGE
%------------------------------------------------

\title[Minimum Water Value]{Stochastic Optimal Control of Renewable Energy:\\
Minimum Water Value Motivation} % The short title appears at the bottom of every slide, the full title is only on the title page

\author{Renzo Caballero} % Your name
\institute[KAUST] % Your institution as it will appear on the bottom of every slide, may be shorthand to save space
{renzo.caballerorosas@kaust.edu.sa\\
\quad\\
\medskip
Advisor: Professor Ra\'ul Tempone\\
}

\begin{document}

\begin{frame}
\titlepage % Print the title page as the first slide
\end{frame}

%------------------------------------------------
%	PRESENTATION SLIDES
%------------------------------------------------

\begin{frame}\frametitle{Bonete (almost 2 years)}
\begin{columns}[c]

\column{.6\textwidth}
\includegraphics[width=1\columnwidth]{bonete.eps}\\

\column{.3\textwidth}
For the past 2 years the water value has changed a lot. We can compute:
\begin{equation*}
\begin{cases}
\hat{\mu}_{\text{bon}}\approx20.6\\
\hat{\sigma}_{\text{bon}}\approx34.1.
\end{cases}
\end{equation*}

\end{columns}
\end{frame}


\begin{frame}\frametitle{Baygorria (almost 2 years)}
\begin{columns}[c]

\column{.6\textwidth}
\includegraphics[width=1\columnwidth]{baygorria.eps}\\

\column{.3\textwidth}
For the past 2 years the water value has been zero:
\begin{equation*}
\begin{cases}
\hat{\mu}_{\text{bay}}\approx0\\
\hat{\sigma}_{\text{bay}}\approx0.
\end{cases}
\end{equation*}

\end{columns}
\end{frame}


\begin{frame}\frametitle{Palmar (almost 2 years)}
\begin{columns}[c]

\column{.6\textwidth}
\includegraphics[width=1\columnwidth]{palmar.eps}\\

\column{.3\textwidth}
For the past 2 years the water value has changed a lot. We can compute:
\begin{equation*}
\begin{cases}
\hat{\mu}_{\text{pal}}\approx20.6\\
\hat{\sigma}_{\text{pal}}\approx34.1.
\end{cases}
\end{equation*}

\end{columns}
\end{frame}


\begin{frame}\frametitle{Salto Grande (almost 2 years)}
\begin{columns}[c]

\column{.6\textwidth}
\includegraphics[width=1\columnwidth]{sg.eps}\\

\column{.3\textwidth}
For the past 2 years the water value has changed a lot. We can compute:
\begin{equation*}
\begin{cases}
\hat{\mu}_{\text{sg}}\approx34.6\\
\hat{\sigma}_{\text{sg}}\approx51.5.
\end{cases}
\end{equation*}

\end{columns}
\end{frame}


\begin{frame}\frametitle{Bonete (2019)}
\begin{columns}[c]

\column{.5\textwidth}
\includegraphics[width=1\columnwidth]{bonete_2019.eps}

\column{.5\textwidth}
\includegraphics[width=0.8\columnwidth]{histBonete.eps}

\end{columns}
\quad\\
\textbf{Here we have $\hat{\mu}_{\text{bon2019}}\approx4.9$ and $\hat{\sigma}_{\text{bon2019}}\approx6.1$.}

\end{frame}


\begin{frame}\frametitle{Baygorria (2019)}
\begin{columns}[c]

\column{.5\textwidth}
\includegraphics[width=1\columnwidth]{baygorria_2019.eps}

\column{.5\textwidth}
\includegraphics[width=0.8\columnwidth]{histBaygorria.eps}

\end{columns}
\quad\\
\textbf{Here we have $\hat{\mu}_{\text{bay2019}}=0$ and $\hat{\sigma}_{\text{bay2019}}=0$.}

\end{frame}


\begin{frame}\frametitle{Palmar (2019)}
\begin{columns}[c]

\column{.5\textwidth}
\includegraphics[width=1\columnwidth]{palmar_2019.eps}

\column{.5\textwidth}
\includegraphics[width=0.8\columnwidth]{histPalmar.eps}

\end{columns}
\quad\\
\textbf{Here we have $\hat{\mu}_{\text{pal2019}}\approx4.9$ and $\hat{\sigma}_{\text{pal2019}}\approx6.0$.}

\end{frame}


\begin{frame}\frametitle{Salto Grande (2019)}
\begin{columns}[c]

\column{.5\textwidth}
\includegraphics[width=1\columnwidth]{sg_2019.eps}

\column{.5\textwidth}
\includegraphics[width=0.8\columnwidth]{histSG.eps}

\end{columns}
\quad\\
\textbf{Here we have $\hat{\mu}_{\text{sg2019}}\approx5.7$ and $\hat{\sigma}_{\text{sg2019}}\approx8.0$.}

\end{frame}


\begin{frame}\frametitle{Conclusion:}
\begin{columns}[c]

\column{.5\textwidth}

\column{.5\textwidth}

\end{columns}
For Uruguay, Bonete and Palmar have almost the same value, Baygorria is always free, and SG has an averaged higher value (and is uncorrelated with the other dams).\\
It does not see it possible to conclude something significant. However, I have noticed that, if for the $i$-th dam with value zero, we choose $1.i\times10^{-3}\ \SI{}{USD/MWh}$ (very small), the system seems to behave well. Also, we can choose the order of the dams in case more than one has value zero.
\end{frame}

\end{document} 