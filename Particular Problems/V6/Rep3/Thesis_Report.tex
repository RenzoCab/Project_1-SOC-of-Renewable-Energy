\PassOptionsToPackage{table}{xcolor}
\documentclass[aspectratio=169]{beamer}\usepackage[utf8]{inputenc}
\usepackage{lmodern}
\usepackage[english]{babel}
\usepackage{color}
\usepackage{amsmath,mathtools}
\usepackage{booktabs}
\usepackage{mathptmx}
\usepackage[11pt]{moresize}
\usepackage{hyperref}
\usepackage{bbm}
\usepackage{subfigure}
\usepackage{siunitx}

\setbeamertemplate{navigation symbols}{}
\setbeamersize{text margin left=5mm,text margin right=5mm}
\setbeamertemplate{caption}[numbered]
\addtobeamertemplate{navigation symbols}{}{
\usebeamerfont{footline}
\usebeamercolor[fg]{footline}
\hspace{1em}
\insertframenumber/\inserttotalframenumber}

\newcommand{\R}{\mathbb{R}}
\newcommand{\E}{\mathbb{E}}
\newcommand{\N}{\mathbb{N}}
\newcommand{\Z}{\mathbb{Z}}
\newcommand{\V}{\mathbb{V}}
\newcommand{\Q}{\mathbb{Q}}
\newcommand{\K}{\mathbb{K}}
\newcommand{\C}{\mathbb{C}}
\newcommand{\T}{\mathbb{T}}
\newcommand{\I}{\mathbb{I}}

\title{Thesis Report}
\subtitle{Renzo Miguel Caballero Rosas}

\begin{document}

\begin{frame}
\titlepage
\end{frame}

\setbeamercolor{background canvas}{bg=white!10}
\begin{frame}\frametitle{Introduction: Connected Dams + Battery}
In this report we will use $\lambda(t)=0$ but a better model for the battery. We will see the effect of having a battery with a quadratic penalization. In this simulation, the penalization when the battery is being used at its maximum power is of the order of the cheapest machine.\\
Next time we will also add a linear penalization such that
\begin{equation*}
C_A(T)=\int_0^T\left(\alpha_1\phi_A(s)+\alpha_2\phi_A^2(s)\right)ds
\end{equation*}
using
\begin{equation*}
\begin{cases}
\alpha_1=\frac{c}{10}\\
\alpha_2=\frac{\alpha_1}{5}\\
\end{cases}
\end{equation*}
where $c$ denotes the cost of the cheapest machine of the system.
\end{frame}

\setbeamercolor{background canvas}{bg=white!10}
\begin{frame}\frametitle{Battery Control:}
\begin{figure}[ht!]
\centering
\subfigure{\includegraphics[scale=0.45]{F3.eps}}
\caption{Now that we have the CFL condition working, we can use a better model for the battery. We choose it such that the energy in the battery never goes below 0 or over 1.}
\end{figure}
\end{frame}

\setbeamercolor{background canvas}{bg=white!10}
\begin{frame}\frametitle{Usage of the battery}
\begin{figure}[ht!]
\centering
\subfigure{\includegraphics[scale=0.45]{F1.eps}}
\subfigure{\includegraphics[scale=0.45]{F2.eps}}
\caption{Adding a quadratic cost reduces the magnitude of the oscillation but they still have a high frequency. We need to add the linear cost and choose better values for tunning.}
\end{figure}
\end{frame}

\setbeamercolor{background canvas}{bg=white!10}
\begin{frame}\frametitle{Costs and Power}
\begin{figure}[ht!]
\centering
\subfigure{\includegraphics[scale=0.42]{F4.eps}}
\subfigure{\includegraphics[scale=0.42]{F5.eps}}
\caption{On the left, we can see the instant costs of each control, the ones that the Hamiltonian optimizes. For the battery, we can see the cost when $\phi_A\approx0$ and when $\phi_A\approx1$ (quadratic cost). On the right the power balance, as we have a quadratic penalization in the battery, it prefers to use small power.}
\end{figure}
\end{frame}

\setbeamercolor{background canvas}{bg=white!10}
\begin{frame}
\begin{figure}[ht!]
\centering
\subfigure{\includegraphics[scale=0.25]{F6.eps}}
\end{figure}
\end{frame}

\setbeamercolor{background canvas}{bg=white!10}
\begin{frame}
\begin{figure}[ht!]
\centering
\subfigure{\includegraphics[scale=0.25]{F7.eps}}
\end{figure}
\end{frame}

\setbeamercolor{background canvas}{bg=white!10}
\begin{frame}\frametitle{I still need to:}
\begin{enumerate}

\item Be sure that the parameters of the dams are correct. I did not verify after adding a condition over the maximum turbine flow, and probably this is why the spillage of Bonete has few effects in Baygorria and all the system.

\item Separate the fuel station into a fuel park using all real data.

\item Add the linear cost to the battery, and choose appropriate values.

\item Program the possibility of ponderation the solution of the HJB equation.

\end{enumerate}
\end{frame}

\end{document}