\PassOptionsToPackage{table}{xcolor}
\documentclass[aspectratio=169]{beamer}\usepackage[utf8]{inputenc}
\usepackage{lmodern}
\usepackage[english]{babel}
\usepackage{color}
\usepackage{amsmath,mathtools}
\usepackage{booktabs}
\usepackage{mathptmx}
\usepackage[11pt]{moresize}
\usepackage{hyperref}
\usepackage{bbm}
\usepackage{subfigure}
\usepackage{siunitx}

\setbeamertemplate{navigation symbols}{}
\setbeamersize{text margin left=5mm,text margin right=5mm}
\setbeamertemplate{caption}[numbered]
\addtobeamertemplate{navigation symbols}{}{
\usebeamerfont{footline}
\usebeamercolor[fg]{footline}
\hspace{1em}
\insertframenumber/\inserttotalframenumber}

\newcommand{\R}{\mathbb{R}}
\newcommand{\E}{\mathbb{E}}
\newcommand{\N}{\mathbb{N}}
\newcommand{\Z}{\mathbb{Z}}
\newcommand{\V}{\mathbb{V}}
\newcommand{\Q}{\mathbb{Q}}
\newcommand{\K}{\mathbb{K}}
\newcommand{\C}{\mathbb{C}}
\newcommand{\T}{\mathbb{T}}
\newcommand{\I}{\mathbb{I}}

\title{Thesis Report}
\subtitle{Renzo Miguel Caballero Rosas}

\begin{document}

\begin{frame}
\titlepage
\end{frame}

\setbeamercolor{background canvas}{bg=white!10}
\begin{frame}\frametitle{Introduction: Connected Dams + Battery}
In this report we will use $\lambda(t)=0$ but a better model for the battery. We will see the effect of having a battery with a very small penalization.
\end{frame}

\setbeamercolor{background canvas}{bg=white!10}
\begin{frame}\frametitle{Battery Control:}
\begin{figure}[ht!]
\centering
\subfigure{\includegraphics[scale=0.45]{F3.eps}}
\caption{Now that we have the CFL condition working, we can use a better model for the battery.}
\end{figure}
\end{frame}

\setbeamercolor{background canvas}{bg=white!10}
\begin{frame}\frametitle{Usage of the battery}
\begin{figure}[ht!]
\centering
\subfigure{\includegraphics[scale=0.45]{F1.eps}}
\subfigure{\includegraphics[scale=0.45]{F2.eps}}
\caption{We can see the oscillations produced for not having a correct penalization.}
\end{figure}
\end{frame}

\setbeamercolor{background canvas}{bg=white!10}
\begin{frame}\frametitle{Costs and Power}
\begin{figure}[ht!]
\centering
\subfigure{\includegraphics[scale=0.45]{F4.eps}}
\subfigure{\includegraphics[scale=0.45]{F5.eps}}
\caption{On the left, we can see the instant costs of each control, the ones that the Hamiltonian optimizes. On the right the power balance, as we have a quadratic penalization in the battery, it prefers to use small power.}
\end{figure}
\end{frame}

\setbeamercolor{background canvas}{bg=white!10}
\begin{frame}
\begin{figure}[ht!]
\centering
\subfigure{\includegraphics[scale=0.3]{F6.eps}}
\end{figure}
\end{frame}

\setbeamercolor{background canvas}{bg=white!10}
\begin{frame}
\begin{figure}[ht!]
\centering
\subfigure{\includegraphics[scale=0.3]{F7.eps}}
\end{figure}
\end{frame}

\end{document}