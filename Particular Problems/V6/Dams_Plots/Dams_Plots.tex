\documentclass[12pt]{article}
\usepackage[table]{xcolor}
\usepackage[margin=1in]{geometry} 
\usepackage{amsmath,amsthm,amssymb}
\usepackage[english]{babel}
\usepackage{tcolorbox}
\usepackage{enumitem}
\usepackage{hyperref}
\usepackage{listings}
\usepackage{blkarray}
\usepackage{float}
\usepackage{bm}
\usepackage{subfigure}
\usepackage{booktabs}
\usepackage[maxfloats=256]{morefloats}
\usepackage{siunitx}
\maxdeadcycles=1000

\setcounter{secnumdepth}{5}
\setcounter{tocdepth}{5}

\newtheorem{theorem}{Theorem}[section]
\newtheorem{corollary}{Corollary}[theorem]
\newtheorem{lemma}[theorem]{Lemma}
\newtheorem{proposition}[theorem]{proposition}
\newtheorem{exmp}{Example}[section]\newtheorem{definition}{Definition}[section]
\newtheorem{remark}{Remark}
\newtheorem{ex}{Exercise}
\theoremstyle{definition}
\theoremstyle{remark}
\bibliographystyle{elsarticle-num}

\DeclareMathOperator{\sinc}{sinc}
\newcommand{\RNum}[1]{\uppercase\expandafter{\romannumeral #1\relax}}
\newcommand{\N}{\mathbb{N}}
\newcommand{\Z}{\mathbb{Z}}
\newcommand{\R}{\mathbb{R}}
\newcommand{\E}{\mathbb{E}}
\newcommand{\matindex}[1]{\mbox{\scriptsize#1}}
\newcommand{\V}{\mathbb{V}}
\newcommand{\Q}{\mathbb{Q}}
\newcommand{\K}{\mathbb{K}}
\newcommand{\C}{\mathbb{C}}
\newcommand{\prob}{\mathbb{P}}

\lstset{numbers=left, numberstyle=\tiny, stepnumber=1, numbersep=5pt}

\begin{document}
\title{Uruguayan Historicals of Production}
\author{Renzo Miguel Caballero Rosas} 
\maketitle

\begin{figure}[ht!]
\centering
\subfigure{\includegraphics[scale=0.7]{1.eps}}
\end{figure}
\begin{figure}[ht!]
\centering
\subfigure{\includegraphics[scale=0.7]{2.eps}}
\end{figure}
\begin{figure}[ht!]
\centering
\subfigure{\includegraphics[scale=0.7]{3.eps}}
\end{figure}
\begin{figure}[ht!]
\centering
\subfigure{\includegraphics[scale=0.7]{4.eps}}
\end{figure}
\begin{figure}[ht!]
\centering
\subfigure{\includegraphics[scale=0.7]{5.eps}}
\end{figure}
\begin{figure}[ht!]
\centering
\subfigure{\includegraphics[scale=0.7]{6.eps}}
\end{figure}
\begin{figure}[ht!]
\centering
\subfigure{\includegraphics[scale=0.7]{7.eps}}
\end{figure}
\begin{figure}[ht!]
\centering
\subfigure{\includegraphics[scale=0.7]{8.eps}}
\end{figure}
\begin{figure}[ht!]
\centering
\subfigure{\includegraphics[scale=0.7]{9.eps}}
\end{figure}
\begin{figure}[ht!]
\centering
\subfigure{\includegraphics[scale=0.7]{10.eps}}
\end{figure}
\begin{figure}[ht!]
\centering
\subfigure{\includegraphics[scale=0.7]{11.eps}}
\end{figure}
\begin{figure}[ht!]
\centering
\subfigure{\includegraphics[scale=0.7]{12.eps}}
\end{figure}
\begin{figure}[ht!]
\centering
\subfigure{\includegraphics[scale=0.7]{13.eps}}
\end{figure}
\begin{figure}[ht!]
\centering
\subfigure{\includegraphics[scale=0.7]{14.eps}}
\end{figure}
\begin{figure}[ht!]
\centering
\subfigure{\includegraphics[scale=0.7]{15.eps}}
\end{figure}
\begin{figure}[ht!]
\centering
\subfigure{\includegraphics[scale=0.7]{16.eps}}
\end{figure}
\begin{figure}[ht!]
\centering
\subfigure{\includegraphics[scale=0.7]{17.eps}}
\end{figure}
\begin{figure}[ht!]
\centering
\subfigure{\includegraphics[scale=0.7]{18.eps}}
\end{figure}

\end{document}