\PassOptionsToPackage{table}{xcolor}
\documentclass[aspectratio=169]{beamer}\usepackage[utf8]{inputenc}
\usepackage{lmodern}
\usepackage[english]{babel}
\usepackage{color}
\usepackage{amsmath,mathtools}
\usepackage{booktabs}
\usepackage{mathptmx}
\usepackage[11pt]{moresize}
\usepackage{hyperref}
\usepackage{bbm}
\usepackage{subfigure}
\usepackage{siunitx}
\usepackage{bm}

\setbeamertemplate{navigation symbols}{}
\setbeamersize{text margin left=5mm,text margin right=5mm}
\setbeamertemplate{caption}[numbered]
\addtobeamertemplate{navigation symbols}{}{
\usebeamerfont{footline}
\usebeamercolor[fg]{footline}
\hspace{1em}
\insertframenumber/\inserttotalframenumber}
\DeclareMathOperator{\tr}{tr}

\newcommand{\R}{\mathbb{R}}
\newcommand{\E}{\mathbb{E}}
\newcommand{\N}{\mathbb{N}}
\newcommand{\Z}{\mathbb{Z}}
\newcommand{\V}{\mathbb{V}}
\newcommand{\Q}{\mathbb{Q}}
\newcommand{\K}{\mathbb{K}}
\newcommand{\C}{\mathbb{C}}
\newcommand{\T}{\mathbb{T}}
\newcommand{\I}{\mathbb{I}}

\title{Thesis Report}
\subtitle{Renzo Miguel Caballero Rosas}

\begin{document}

\begin{frame}
\titlepage
\end{frame}

\setbeamercolor{background canvas}{bg=white!10}
\begin{frame}\frametitle{Introduction: Connected Dams + Battery}
In this simulation we used $\lambda(t)=0$, $2^2$ discretizations for the dams, $2^5$ for the battery and $2^{11}$ for the time (total simulation time: 2 hrs.). For computing the optimal path, we do linear interpolation in space increasing the number of discretizations five times in each direction.\\
We have four dams (with 3 of them connected), four fuel stations and a battery.
\begin{figure}[ht!]
\centering
\subfigure{\includegraphics[scale=0.3]{1.png}}
\end{figure}
\end{frame}

\setbeamercolor{background canvas}{bg=white!10}
\begin{frame}\frametitle{CFL Condition:}
Finite time horizon problem: Let us consider a control model in which the state evolves according to an $\R^n$-values process $x(s)$ governed by a system of stochastic differential equations of the form
\begin{equation*}
dx=f(s,x(s),u(s))ds+\sigma(s,x(s),u(s))dw(s),\ t\leq s\leq t_1,
\end{equation*}
where $u(s)\in U$ is the control applied at time $s$ and $w(s)$ is a Brownian motion of dimension $d$. Also we have the cost 
\begin{equation*}
J(t,x;u)=\E_{tx}\left\{\int_t^\tau L(s,x(s),u(s))ds+\Psi(\tau,x(\tau))\right\}.
\end{equation*}
\end{frame}

\setbeamercolor{background canvas}{bg=white!10}
\begin{frame}\frametitle{CFL Condition:}
Let $S_+^n$ denote the set of symmetric, non-negative definite $n\times n$ matrices $\bm{A}=(\bm{A}_{ij})$ with $i,j=1,\dots,n$. Let $a=\sigma\sigma'$ and
\begin{equation*}
\tr a\bm{A}=\sum_{i,j=1}^na_{ij}\bm{A}_{ij}.
\end{equation*}
For $(t,x)\in\overline{Q}_0,p\in\R^n,\bm{A}\in S^n_+$, let
\begin{equation*}
\mathcal{H}(t,x,p,\bm{A})=\sup_{v\in U}\left[-f(t,x,v)\cdot p-\frac{1}{2}\tr a(t,x,v)\bm{A}-L(t,x,v)\right].
\end{equation*}
Then the Hamilton-Jacobi-Bellman partial differential equation associated with this optimal stochastic control problem is given by
\begin{equation*}
-\frac{\partial V}{\partial t}+\mathcal{H}(t,x,D_xV,D^2_xV)=0,\ (t,x)\in Q.
\end{equation*}
\end{frame}

\setbeamercolor{background canvas}{bg=white!10}
\begin{frame}\frametitle{CFL Condition:}
We take autonomous $f(x,v),\sigma(x,v)$ and $L(x,v)$, where now $x\in\R^n,f=(f_1,\dots,f_n)$ is $\R^n$-valued and $a=\sigma\sigma'$ is $n\times n$-matrix valued. The matrices $a(x,v)=(a_{ij}(x,v)),i,j=1,\dots,n$, are non-negative definite. Hence $a_{ij}\geq0$. Given the time step $h>0$ and the spatial step $\delta>0$, and the following condition
\begin{equation*}
a_{ij}(x,v)-\sum_{j\neq i}|a_{ij}(x,v)|\geq0,
\end{equation*}
we need to ensure the next condition\begin{equation*}
h\sum_{i=1}^n\left[a_{ii}(x,v)-\frac{1}{2}\sum_{j\neq i}|a_{ij}(x,v)|+\delta|f_i(x,v)|\right]\leq\delta^2.
\end{equation*}
In the case without diffusion ($a_{ij}=0$) and different spatial discretizations $\{\delta_i\}_{i=1}^n$,
\end{frame}

\setbeamercolor{background canvas}{bg=white!10}
\begin{frame}\frametitle{CFL Condition:}
the previous condition is given by
\begin{equation*}
\boxed{h\leq\frac{1}{\sum_{i=1}^n\left|\frac{f_i(x,v)}{\delta_i}\right|}}
\end{equation*}
\end{frame}

\setbeamercolor{background canvas}{bg=white!10}
\begin{frame}\frametitle{Battery Controls:}
\begin{figure}[ht!]
\centering
\subfigure{\includegraphics[scale=0.45]{11.eps}}
\caption{We choose this bounds over the control in a way such that always $A\in[0,1]$.}
\end{figure}
\end{frame}

\setbeamercolor{background canvas}{bg=white!10}
\begin{frame}\frametitle{Controls:}
\begin{figure}[ht!]
\centering
\subfigure{\includegraphics[scale=0.45]{7.eps}}
\subfigure{\includegraphics[scale=0.45]{8.eps}}
\caption{Here we can see all the controls but the battery one. Is important to notice that the total time is 24 hrs.}
\end{figure}
\end{frame}

\setbeamercolor{background canvas}{bg=white!10}
\begin{frame}\frametitle{Battery controls:}
\begin{figure}[ht!]
\centering
\subfigure{\includegraphics[scale=0.45]{2.eps}}
\subfigure{\includegraphics[scale=0.45]{13.eps}}
\caption{To make smoother the control of the battery is possible. However, too many restrictions over this control limit the possibility of the battery to get totally charged and discharged. In the battery, we are only penalizing the first derivative (adding linear and quadratic costs did not improve the osculations.)}
\end{figure}
\end{frame}

\setbeamercolor{background canvas}{bg=white!10}
\begin{frame}\frametitle{Energy and Costs:}
\begin{figure}[ht!]
\centering
\subfigure{\includegraphics[scale=0.45]{4.eps}}
\subfigure{\includegraphics[scale=0.45]{5.eps}}
\caption{In the case of the energy, we only consider when the battery provides power (otherwise its total influence would be 0). For the costs, we avoid the battery.}
\end{figure}
\end{frame}

\setbeamercolor{background canvas}{bg=white!10}
\begin{frame}\frametitle{Instant costs:}
\begin{figure}[ht!]
\centering
\subfigure{\includegraphics[scale=0.32]{1.eps}}
\subfigure{\includegraphics[scale=0.32]{9.eps}}
\subfigure{\includegraphics[scale=0.32]{10.eps}}
\caption{Here we can see the value of each source over the optimal path (and over time). The oscillations in the battery's control are produced by its instant costs (which has small oscillations). Notice that for the turbine flow, the final value (at $t=1$) is the water's value.}
\end{figure}
\end{frame}

\setbeamercolor{background canvas}{bg=white!10}
\begin{frame}\frametitle{Demand:}
\begin{figure}[ht!]
\centering
\subfigure{\includegraphics[scale=0.45]{3.eps}}
\subfigure{\includegraphics[scale=0.45]{18.eps}}
\caption{Again, tunning the penalization of the battery we can get smoother controls. However, we must check the specifications of the battery.}
\end{figure}
\end{frame}

\setbeamercolor{background canvas}{bg=white!10}
\begin{frame}\frametitle{Demand:}
\begin{figure}[ht!]
\centering
\subfigure{\includegraphics[scale=0.45]{19.eps}}
\subfigure{\includegraphics[scale=0.45]{20.eps}}
\caption{More views.}
\end{figure}
\end{frame}

\setbeamercolor{background canvas}{bg=red!20}
\begin{frame}\frametitle{I still need to:}
\begin{enumerate}
\item Be sure I am computing well the CFL condition; I have the feeling that I have a bug, but I still did not check it carefully.
\item Be sure that the parameters of the dams are correct. I did not verify after adding a
condition over the maximum turbine flow, and probably this is why the spillage of
Bonete has few effects in Baygorria and all the system.
\item Make the algorithm adaptative over the time (change $\Delta t$ depending on the CFL condition). Until now I am just using a small $\Delta t$.
\item Compute the interpolation using multilinear interpolation (and check Transfinite interpolation)
\item Find a real model for the control of the battery.
\item Use sub-gradients to compute $\lambda^*(t)$.
\item Make parallel the Finite Differences part.
\item Check the jump in Baygorria's control.
\end{enumerate}
\end{frame}

\end{document}