\PassOptionsToPackage{table}{xcolor}
\documentclass[aspectratio=169]{beamer}\usepackage[utf8]{inputenc}
\usepackage{lmodern}
\usepackage[english]{babel}
\usepackage{color}
\usepackage{amsmath,mathtools}
\usepackage{booktabs}
\usepackage{mathptmx}
\usepackage[11pt]{moresize}
\usepackage{hyperref}
\usepackage{bbm}
\usepackage{subfigure}
\usepackage{siunitx}

\setbeamertemplate{navigation symbols}{}
\setbeamersize{text margin left=5mm,text margin right=5mm}
\setbeamertemplate{caption}[numbered]
\addtobeamertemplate{navigation symbols}{}{
\usebeamerfont{footline}
\usebeamercolor[fg]{footline}
\hspace{1em}
\insertframenumber/\inserttotalframenumber}

\newcommand{\R}{\mathbb{R}}
\newcommand{\E}{\mathbb{E}}
\newcommand{\N}{\mathbb{N}}
\newcommand{\Z}{\mathbb{Z}}
\newcommand{\V}{\mathbb{V}}
\newcommand{\Q}{\mathbb{Q}}
\newcommand{\K}{\mathbb{K}}
\newcommand{\C}{\mathbb{C}}
\newcommand{\T}{\mathbb{T}}
\newcommand{\I}{\mathbb{I}}

\title{Thesis Report}
\subtitle{Renzo Miguel Caballero Rosas}

\begin{document}

\begin{frame}
\titlepage
\end{frame}

\setbeamercolor{background canvas}{bg=white!10}
\begin{frame}\frametitle{Setup: Electric Model}
Our model includes:
\begin{enumerate}
\item A sinusoidal demand given by $(\SI{2e6}{kW})\left(0.8+0.3\sin\left(\frac{\pi}{7t}\right)\right)$. The idea is to use a demand that forces the system to pass through a lot of different cases.
\item Normalized time ($t\in[0,1]$) with a real window from 00:00 hrs to 24:00 hrs. We call to 00:00 hrs the initial time and 24:00 hrs the final time.
\item As generators: the dams Salto Grande and Bonete, wind power and a fuel station.
\end{enumerate}
\end{frame}

\setbeamercolor{background canvas}{bg=white!10}
\begin{frame}\frametitle{Setup: Numerical Method}
\textbf{Minimization}: to minimize the Hamiltonian we use a framework consisting in increasing the restrictions over the minimum until we find a admissible solution (one that satisfy all the restrictions). Every some amount of optimizations we compare our solution with the result of \emph{fmincon}. Now if working for two dams, but it should be working for any number of dams.\\
\textbf{Scheme}: We use finite differences in all space direction, and forward Euler in time. For each point in our grid we check monotonicity of our scheme using as optimal controls the ones in the last time step (we assume the controls do not change too much for small time steps). Our scheme has adaptivity, given some condition we chose between left-hand or right-hand finite differences to ensure the monotonicity.
\end{frame}

\setbeamercolor{background canvas}{bg=white!10}
\begin{frame}\frametitle{Numerical Parameters: Simulation with 4 values of wind}
\begin{table}[]
\begin{tabular}{ccccc}
\toprule
Simulation & $N_T$ & $N_{V_1}$ & $N_{V_4}$ & $N_W$ \\ \midrule
1 & 4 & 4 & 4 & 4 \\
2 & 8 & 8 & 8 & 4 \\
3 & 16 & 16 & 16 & 4 \\
4 & 32 & 32 & 32 & 4 \\
5 & 64 & 64 & 64 & 4 \\
6 & 128 & 128 & 128 & 4 \\
7 & 256 & 256 & 256 & 4 \\ \bottomrule
\end{tabular}
\caption{Number of discretizations in space and time for each simulation.}
\label{Table_1}
\end{table}
\end{frame}

\setbeamercolor{background canvas}{bg=white!10}
\begin{frame}\frametitle{Numerical Parameters: Simulation with 8 values of wind}
\begin{table}[]
\begin{tabular}{ccccc}
\toprule
Simulation & $N_T$ & $N_{V_1}$ & $N_{V_4}$ & $N_W$ \\ \midrule
8 & 4 & 4 & 4 & 8 \\
9 & 8 & 8 & 8 & 8 \\
10 & 16 & 16 & 16 & 8 \\
11 & 32 & 32 & 32 & 8 \\
12 & 64 & 64 & 64 & 8 \\
13 & 128 & 128 & 128 & 8 \\
14 & 256 & 256 & 256 & 8 \\ \bottomrule
\end{tabular}
\caption{Number of discretizations in space and time for each simulation.}
\label{Table_2}
\end{table}
\end{frame}

\setbeamercolor{background canvas}{bg=white!10}
\begin{frame}\frametitle{Results: Optimization}
We compare with \emph{fmincon} every 1000 optimizations. For the simulation 5 (which has a grid with 1373125 points), the maximum absolute error between the controls in the optimizations were $\num{5.4e-3}$, $\num{8.07e-4}$ and $\num{1.94e-16}$ for $\phi_T^{(1)}$, $\phi_T^{(4)}$ and $\phi_F$ respectively (all controls are between 0 and 1). All the next results are from the simulation 5, except the convergence tests.
\end{frame}

\setbeamercolor{background canvas}{bg=white!10}
\begin{frame}\frametitle{Results: Effective demand curves}
\begin{figure}[h!]
\centering
\includegraphics[width=0.6\textwidth]{Demand.eps}
\caption{The effective demand for each value of wind, its the demand minus the power generated by that value of wind. We can see five different values of wind (0.3,0.4,0.5,0.6,0.7). The value of the wind is normalized between 0 and 1.}
\end{figure}
\end{frame}

\setbeamercolor{background canvas}{bg=white!10}
\begin{frame}\frametitle{Results: Cost function and Hamiltonian}
\begin{figure}[ht!]
\centering
\subfigure
{\includegraphics[scale=.2]{Cost_Function.eps}}
\qquad
\subfigure
{\includegraphics[scale=.2]{Hamiltonian.eps}}
\caption
{(a) Optimal cost function for the initial time. (b) Hamiltonian for the initial time.}
\end{figure}
\end{frame}

\setbeamercolor{background canvas}{bg=white!10}
\begin{frame}\frametitle{Results: Derivatives with respect to the volumes of water}
\begin{figure}[ht!]
\centering
\subfigure
{\includegraphics[scale=.2]{uv1.eps}}
\qquad
\subfigure
{\includegraphics[scale=.2]{uv4.eps}}
\caption
{(a) Derivative with respect to $V^{(1)}$. (b) Derivative with respect to $V^{(2)}$.}
\end{figure}
\end{frame}

\setbeamercolor{background canvas}{bg=white!10}
\begin{frame}\frametitle{Results: Derivatives with respect to wind}
\begin{figure}[ht!]
\centering
\subfigure
{\includegraphics[scale=.2]{uw.eps}}
\qquad
\subfigure
{\includegraphics[scale=.2]{uww.eps}}
\caption
{(a) Derivative with respect to $W$. (b) Second derivative with respect to $W$.}
\end{figure}
\end{frame}

\setbeamercolor{background canvas}{bg=white!10}
\begin{frame}\frametitle{Results: Cost function at the initial point $\left(V^{(1)}_0,V^{(4)}_0\right)$}
\begin{figure}[h!]
\centering
\includegraphics[width=0.6\textwidth]{CvsW.eps}
\caption{Cost function for volumes fixed in $\left(V^{(1)}_0,V^{(4)}_0\right)$, as a function of the wind and time. At final time the cost is 0 and for earlier times, the cost increases.}
\end{figure}
\end{frame}

\setbeamercolor{background canvas}{bg=white!10}
\begin{frame}\frametitle{Results: Convergence tests}
\begin{figure}[ht!]
\centering
\subfigure
{\includegraphics[scale=.5]{Relative_error.eps}}
\qquad
\subfigure
{\includegraphics[scale=.45]{Cost_function_at_inicial_point.eps}}
\caption
{(a) Relative error. (b) Value of the cost function at the initial point.}
\end{figure}
\end{frame}

\end{document}