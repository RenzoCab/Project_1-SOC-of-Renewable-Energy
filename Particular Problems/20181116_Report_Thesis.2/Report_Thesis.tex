\PassOptionsToPackage{table}{xcolor}
\documentclass[aspectratio=169]{beamer}\usepackage[utf8]{inputenc}
\usepackage{lmodern}
\usepackage[english]{babel}
\usepackage{color}
\usepackage{amsmath,mathtools}
\usepackage{booktabs}
\usepackage{mathptmx}
\usepackage[11pt]{moresize}
\usepackage{hyperref}
\usepackage{bbm}
\usepackage{subfigure}
\usepackage{siunitx}

\setbeamertemplate{navigation symbols}{}
\setbeamersize{text margin left=5mm,text margin right=5mm}
\setbeamertemplate{caption}[numbered]
\addtobeamertemplate{navigation symbols}{}{
\usebeamerfont{footline}
\usebeamercolor[fg]{footline}
\hspace{1em}
\insertframenumber/\inserttotalframenumber}

\newcommand{\R}{\mathbb{R}}
\newcommand{\E}{\mathbb{E}}
\newcommand{\N}{\mathbb{N}}
\newcommand{\Z}{\mathbb{Z}}
\newcommand{\V}{\mathbb{V}}
\newcommand{\Q}{\mathbb{Q}}
\newcommand{\K}{\mathbb{K}}
\newcommand{\C}{\mathbb{C}}
\newcommand{\T}{\mathbb{T}}
\newcommand{\I}{\mathbb{I}}

\title{Thesis Report:\\
System with two connected dams and one fuel station}
\subtitle{Renzo Miguel Caballero Rosas}

\begin{document}

\begin{frame}
\titlepage
\end{frame}

\setbeamercolor{background canvas}{bg=white!10}
\begin{frame}\frametitle{System:}
We have Bonete and Salto Grande, where all the water from Bonete reachs Salto Grande after 8 hrs. We use a Lagrangian relaxation to avoid the no markovianity relation on the system. As important values we have the cost for using the normalized controls:
\begin{enumerate}
\item Bonete in Cost function: $\num{9.5096e-4}$.
\item Salto Grande in Cost function: $\num{6.25e-4}$.
\item Fuel Station in Cost function: $6.24$.
\end{enumerate}
And the costs of the energies:
\begin{enumerate}
\item Bonete power cost: $\SI{3e-07}{MUSD/MWh}$.
\item Salto Grande power cost: $\SI{3e-08}{MUSD/MWh}$.
\item Fuel Station power cost: $\SI{3.12e-3}{MUSD/MWh}$.
\end{enumerate}
We set Salto Grande as the cheaper generator, to simulate the case between Bonete and Baygorria.
\end{frame}

\setbeamercolor{background canvas}{bg=white!10}
\begin{frame}\frametitle{Results: With $\lambda(t)=A$}
\begin{figure}[h!]
\centering
\includegraphics[width=0.6\textwidth]{Cost_Lambda_0.eps}
\caption{Solution without the relaxation is 1.3225 MUSD while the maximum with relaxation is 1.4709 MUSD. The maximum is over the red line and is for a negative value of $\lambda(t)$.}
\end{figure}
\end{frame}

\setbeamercolor{background canvas}{bg=white!10}
\begin{frame}\frametitle{Results: With $\lambda(t)=A+B\sin(3\pi t)$}
\begin{figure}[h!]
\centering
\includegraphics[width=0.6\textwidth]{Cost_Lambda_1.eps}
\caption{Solution without the relaxation is 1.3225 MUSD while the maximum with relaxation is 1.4256 MUSD.}
\end{figure}
\end{frame}

\setbeamercolor{background canvas}{bg=white!10}
\begin{frame}\frametitle{Results: With $\lambda(t)=A_0+\sum_{i=1}^{10}\left[A_i\sin(2\pi i t)+B_i\cos(2\pi it)\right]$}
\begin{figure}[ht!]
\centering
\subfigure
{\includegraphics[scale=.4]{Cost_Lambda_2.eps}}
\qquad
\subfigure
{\includegraphics[scale=.4]{Cost_Lambda_2_OP.eps}}
\caption
{(a) Optimal $\lambda(t)$ function, it takes positive and negative values over the time. (b) Convergence to the minimum using \emph{fminsearch}, the minimum without relaxation is -1.2808 MUSD.}
\end{figure}
\end{frame}

\setbeamercolor{background canvas}{bg=white!10}
\begin{frame}\frametitle{Results: With $\lambda(t)=\sum_{i=0}^{20}A_it^i$}
\begin{figure}[ht!]
\centering
\subfigure
{\includegraphics[scale=.4]{Cost_Lambda_3.eps}}
\qquad
\subfigure
{\includegraphics[scale=.4]{Cost_Lambda_3_OP.eps}}
\caption
{(a) Optimal $\lambda(t)$ function, it takes positive and negative values over the time. (b) Convergence to the minimum using \emph{fminsearch}, the minimum without relaxation is -1.2808 MUSD.}
\end{figure}
\end{frame}

\setbeamercolor{background canvas}{bg=white!10}
\begin{frame}\frametitle{Results: With $\lambda(t)=\sum A_i\delta(t-i)$}
\begin{figure}[ht!]
\centering
\subfigure
{\includegraphics[scale=.4]{Cost_Lambda_4.eps}}
\qquad
\subfigure
{\includegraphics[scale=.4]{Cost_Lambda_4_OP.eps}}
\caption
{(a) Optimal $\lambda(t)$ function, it takes positive and negative values over the time. (b) Convergence to the minimum using \emph{fminsearch}, the minimum without relaxation is -1.2808 MUSD.}
\end{figure}
\end{frame}

\setbeamercolor{background canvas}{bg=white!10}
\begin{frame}\frametitle{Results: Conclusions}
\begin{enumerate}
\item It is not possible to know with accuracy the value of the $\lambda(t)$ function because it has mixed effects over the system. This mixed effects can be seen with more detail in the Lagrangian Multipliers file.
\item Also to add an extra control not always results into a reduction of the final cost, this is again by the mixed effects.
\end{enumerate}

\end{frame}

\end{document}