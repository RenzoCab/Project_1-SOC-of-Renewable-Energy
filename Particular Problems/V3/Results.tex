\PassOptionsToPackage{table}{xcolor}
\documentclass[aspectratio=169]{beamer}\usepackage[utf8]{inputenc}
\usepackage{lmodern}
\usepackage[english]{babel}
\usepackage{color}
\usepackage{amsmath,mathtools}
\usepackage{booktabs}
\usepackage{mathptmx}
\usepackage[11pt]{moresize}
\usepackage{hyperref}
\usepackage{bbm}
\usepackage{subfigure}

\setbeamertemplate{navigation symbols}{}
\setbeamersize{text margin left=5mm,text margin right=5mm}
\setbeamertemplate{caption}[numbered]
\addtobeamertemplate{navigation symbols}{}{
\usebeamerfont{footline}
\usebeamercolor[fg]{footline}
\hspace{1em}
\insertframenumber/\inserttotalframenumber}

\newcommand{\R}{\mathbb{R}}
\newcommand{\E}{\mathbb{E}}
\newcommand{\N}{\mathbb{N}}
\newcommand{\Z}{\mathbb{Z}}
\newcommand{\V}{\mathbb{V}}
\newcommand{\Q}{\mathbb{Q}}
\newcommand{\K}{\mathbb{K}}
\newcommand{\C}{\mathbb{C}}
\newcommand{\T}{\mathbb{T}}
\newcommand{\I}{\mathbb{I}}

\title{Problem 1: Results}
\subtitle{Renzo Miguel Caballero Rosas}

\begin{document}

\begin{frame}
\titlepage
\end{frame}

\setbeamercolor{background canvas}{bg=white!10}
\begin{frame}\frametitle{Case 1}
Here we take $K_{HE}=0.3$ U\$S/MWh; it is converted to $K_H=1.8*10^{-5}$ U\$S/m$^3$. As is it too much smaller than the price per energy of the fuel, the system tries to use water always it can.
\end{frame}

\setbeamercolor{background canvas}{bg=white!10}
\begin{frame}\frametitle{Case 1}
\begin{figure}[ht!]
\centering
\subfigure{\includegraphics[scale=.5]{Result_1.eps}}
\subfigure{\includegraphics[scale=.5]{DU_1.eps}}
\caption{Cost function and gradient. The more water we have, the less the cost function increases over time.}
\end{figure}
\end{frame}

\setbeamercolor{background canvas}{bg=white!10}
\begin{frame}\frametitle{Case 1}
\begin{figure}[ht!]
\centering
\subfigure{\includegraphics[scale=.5]{PhiT_1.eps}}
\subfigure{\includegraphics[scale=.5]{PhiF_1.eps}}
\caption{Controls of the dam and fuel station.}
\end{figure}
\end{frame}

\setbeamercolor{background canvas}{bg=white!10}
\begin{frame}\frametitle{Case 1}
\begin{figure}[ht!]
\centering
\subfigure{\includegraphics[scale=.5]{D5000_1.eps}}
\subfigure{\includegraphics[scale=.5]{D3000_1.eps}}
\caption{Demand and power supply in time. As we can see, the water's power follows the demand as much as it can, up to its limit.}
\end{figure}
\end{frame}

\end{document}