\documentclass[12pt]{article}
\usepackage[table]{xcolor}
\usepackage[margin=1in]{geometry} 
\usepackage{amsmath,amsthm,amssymb}
\usepackage[english]{babel}
\usepackage{tcolorbox}
\usepackage{enumitem}
\usepackage{hyperref}
\usepackage{listings}
\usepackage{bbm}
\usepackage{blkarray}
\usepackage{float}
\usepackage{bm}
\usepackage{subfigure}
\usepackage{booktabs}

\setcounter{secnumdepth}{5}
\setcounter{tocdepth}{5}

\newtheorem{theorem}{Theorem}[section]
\newtheorem{corollary}{Corollary}[theorem]
\newtheorem{lemma}[theorem]{Lemma}
\newtheorem{proposition}[theorem]{proposition}
\newtheorem{exmp}{Example}[section]\newtheorem{definition}{Definition}[section]
\newtheorem{remark}{Remark}
\newtheorem{ex}{Exercise}
\theoremstyle{definition}
\theoremstyle{remark}
\bibliographystyle{elsarticle-num}

\DeclareMathOperator{\sinc}{sinc}
\newcommand{\RNum}[1]{\uppercase\expandafter{\romannumeral #1\relax}}
\newcommand{\N}{\mathbb{N}}
\newcommand{\Z}{\mathbb{Z}}
\newcommand{\R}{\mathbb{R}}
\newcommand{\E}{\mathbb{E}}
\newcommand{\matindex}[1]{\mbox{\scriptsize#1}}
\newcommand{\V}{\mathbb{V}}
\newcommand{\Q}{\mathbb{Q}}
\newcommand{\K}{\mathbb{K}}
\newcommand{\C}{\mathbb{C}}
\newcommand{\prob}{\mathbb{P}}

\lstset{numbers=left, numberstyle=\tiny, stepnumber=1, numbersep=5pt}

\begin{document}
\title{Problem 1: 1 Dam + 1 Fuel Generator (all deterministic), non linearity in the dam.}
\author{Renzo Miguel Caballero Rosas} 
\maketitle

\pagebreak
\tableofcontents
\pagebreak

\section{Model}

Here we have Salto Grande and one fuel generator, also the demand is deterministic. We use time in seconds, power in kW, money in U\$S, distance in meters and volume in $m^3$.

\subsection{Hydraulic power}

The dynamics of the system are given by
\begin{equation*}
\begin{cases}
dV(t)=(I_T(t)-\phi_T(t))dt\\
V(t_0)=5000\ hm^3=5.000*10^9\ m^3\\
3.0*10^9\leq V(t)\leq5.0*10^9\ m^3,\\
0\leq\phi(t)\leq4200\ m^3/s,
\end{cases}
\end{equation*}
the power and cost functions are
\begin{equation*}
\begin{cases}
P_H(t)=\eta\phi_T(t)h(t)=8.72\phi_T(t)\Big[H(V(t))-dh(\phi_T(t))-5.1\Big]\\
H(V)=(-7.91*10^{-19})V^2+(1.03*10^{-8})V+3.79\\
dh(\phi_T)=(1.27*10^{-3})\phi_T
\end{cases}
\end{equation*}
and
\begin{equation*}
C_H(t)=K_H\int_0^t\phi_T(s)ds=1.8*10^{-5}\int_0^t\phi_T(s)ds.
\end{equation*}
We set $I_T(t)=2.2*10^3\ m^3/s$, corresponding to half of the maximum turbined flow.

\subsection{Fossil fuel power station}

The power and cost functions are
\begin{equation*}
\begin{cases}
P_F(t)=\phi_F(t)\\
0\leq\phi_F(t)\leq\overline{\phi_F}=2.0*10^6\ kW
\end{cases}
\end{equation*}
and
\begin{equation*}
C_F(t)=\frac{K_F}{3.6*10^6}\int_0^tP_F(s)ds=(3.6*10^{-5})\int_0^t\phi_F(s)ds.
\end{equation*}

\subsection{Demand}

Deterministic and only depending on time $D(t)$, we will use $D(t)=1000+500\sin\big(\frac{2\pi}{12*3600}t\big)$ (Figure (\ref{D_1})). Also we use $D(t)=1000$ MW constant.
%\begin{figure}[ht!]
%\centering
%\includegraphics[width=0.5\textwidth]{Demand_01012017.eps}
%\caption{Demand of 01/01/2017 (from \href{adme.com.uy}{adme.com.uy}).}
%\label{Demand01012017}
%\end{figure}
\begin{figure}[ht!]
\centering
\includegraphics[width=0.5\textwidth]{D_1.eps}
\caption{Synthetic demand.}
\label{D_1}
\end{figure}

\subsection{Constraint}

We need at any moment to supply the demand, this is
\begin{equation*}
D(t)=P_H(t)+P_F(t).
\end{equation*}

\subsection{Objective function}

We want to minimize
\begin{equation*}
\Theta(T)=\E\Big[C_H(T)+C_F(T)\Big]=K_H\int_0^T\phi_H(s)ds+\frac{K_F}{3.6*10^6}\int_0^T\phi_F(s)ds.
\end{equation*}

\section{Construction of the H-J-B equation}

\subsection{Hamiltonian}

Our Hamiltonian is given by
\begin{multline*}
\mathbf{H}(Du,u,t,V)=\\
=\min_{\begin{cases}
0\leq\phi_H(t)\leq\overline{\phi_T}\\
0\leq\phi_F(t)\leq\overline{\phi_F}\\
D(t)=P_H(t)+P_F(t)
\end{cases}}\Bigg[(I_T-\phi_T)\frac{\partial u}{\partial V}+\frac{K_F}{3.6*10^6}\phi_F+K_H\phi_T\Bigg]\\
=\min_{\begin{cases}
0\leq\phi_H(t)\leq\overline{\phi_T}\\
0\leq\phi_F(t)\leq\overline{\phi_F}\\
D(t)=P_H(t)+P_F(t)
\end{cases}}\Bigg[-\phi_T\frac{\partial u}{\partial V}+\frac{K_F}{3.6*10^6}\phi_F+K_H\phi_T\Bigg]+I_T\frac{\partial u}{\partial V}\\
=\min_{\begin{cases}
0\leq\phi_H(t)\leq\overline{\phi_T}\\
0\leq\phi_F(t)\leq\overline{\phi_F}\\
D(t)=P_H(t)+P_F(t)
\end{cases}}\Bigg[\Bigg(-\frac{\partial u}{\partial V}+K_H\Bigg)\phi_H+\Bigg(\frac{K_F}{3.6*10^6}\Bigg)\phi_F\Bigg]+I_T\frac{\partial u}{\partial V}.
\end{multline*}
%We have that $\underline{H}=30$ and $\overline{H}=35.5$, then $h(t)=(H(V(t))-5.1)\in[24.9,30.4]$ for all $t\geq0$. Now we compare the influence of $\phi_T$ and $\phi_F$, it is natural to assume $K_F>-\frac{\partial u}{\partial V}+K_H\eta(H(V)-5.1)$ which is correct if $\frac{\partial u}{\partial V}>K_H\eta(H(V)-5.1)-K_F(>-50.5)$.

\subsection{H-J equation}

The Hamilton-Jacobi equation is
\begin{equation*}
\begin{cases}
\frac{\partial u}{\partial t}+\mathbf{H}(Du,u,t,V)=0\\
u(T,v)=0.
\end{cases}
\end{equation*}

\section{Simulation}

\subsection{Scheme and discretization}

We use central finite differences and explicit Euler. For the volume, we simulate between 3000 and 5000 $hm^3$ with $dV=10\ hm^3$, and for the time, we simulate from 24:00 to 00:00 hs (86400 to 0 seconds) with $dt=1200$ seconds. For the demand, we use the one defined before (see Figure (\ref{D_1})). The boundary conditions are $\frac{\partial^2 u}{\partial V^2}=0$ in both extremes.

\subsection{Results}

The results are in the presentation.

%\subsubsection{Case 1}
%
%Here we take $K_{HE}=0.3$ U\$S/MWh. As is it too much smaller than the price per energy of the fuel, the system try to use water always it can. The figures are from (\ref{Result_1}) to (\ref{D3000_1}).
%\begin{figure}[ht!]
%\centering
%\includegraphics[width=0.7\textwidth]{Result_1.eps}
%\caption{Optimal cost function.}
%\label{Result_1}
%\end{figure}
%\begin{figure}[ht!]
%\centering
%\includegraphics[width=0.7\textwidth]{DU_1.eps}
%\caption{Gradient function.}
%\label{DU_1}
%\end{figure}
%\begin{figure}[ht!]
%\centering
%\includegraphics[width=0.7\textwidth]{PhiT_1.eps}
%\caption{Turbined flow.}
%\label{PhiT_1}
%\end{figure}
%\begin{figure}[ht!]
%\centering
%\includegraphics[width=0.7\textwidth]{PhiF_1.eps}
%\caption{Fuel control.}
%\label{PhiF_1}
%\end{figure}
%\begin{figure}[ht!]
%\centering
%\includegraphics[width=0.7\textwidth]{D5000_1.eps}
%\caption{Demand and all the powers with $V=5000\ hm^3$.}
%\label{D5000_1}
%\end{figure}
%\begin{figure}[ht!]
%\centering
%\includegraphics[width=0.7\textwidth]{D3000_1.eps}
%\caption{Demand and all the powers with $V=3000\ hm^3$.}
%\label{D3000_1}
%\end{figure}
%
%\subsubsection{Case 2}
%
%Here we take $K_{HE}=130$ U\$S/MWh. Also the price is higher when the dam start to have less water, then the system changes between water and fuel. The figures are from (\ref{Result_2}) to (\ref{D3000_2}).
%\begin{figure}[ht!]
%\centering
%\includegraphics[width=0.7\textwidth]{Result_2.eps}
%\caption{Optimal cost function.}
%\label{Result_2}
%\end{figure}
%\begin{figure}[ht!]
%\centering
%\includegraphics[width=0.7\textwidth]{DU_2.eps}
%\caption{Gradient function.}
%\label{DU_2}
%\end{figure}
%\begin{figure}[ht!]
%\centering
%\includegraphics[width=0.7\textwidth]{PhiT_2.eps}
%\caption{Turbined flow.}
%\label{PhiT_2}
%\end{figure}
%\begin{figure}[ht!]
%\centering
%\includegraphics[width=0.7\textwidth]{PhiF_2.eps}
%\caption{Fuel control.}
%\label{PhiF_2}
%\end{figure}
%\begin{figure}[ht!]
%\centering
%\includegraphics[width=0.7\textwidth]{D5000_2.eps}
%\caption{Demand and all the powers with $V=5000\ hm^3$.}
%\label{D5000_2}
%\end{figure}
%\begin{figure}[ht!]
%\centering
%\includegraphics[width=0.7\textwidth]{D3000_2.eps}
%\caption{Demand and all the powers with $V=3000\ hm^3$.}
%\label{D3000_2}
%\end{figure}


\section{Extra}

Here we have some computations and assumptions.

\subsection{Water's value}

Our data can be the value of the water's energy, this is $K_{HE}$ in $U\$S/MWh$, but we need the value per volume $K_H$ in $U\$S/m^3$, then we need to do a conversion. Assuming maximum volume and that we use the dam at its maximum power, we want
\begin{equation*}
\frac{K_{HE}}{1000}\int_0^TP_H(s)ds=K_H\int_0^T\phi_T(s)ds.
\end{equation*}
Now, in $T=1$ hs we have
\begin{equation*}
\begin{cases}
\frac{K_{HE}}{1000}\int_0^TP_H(s)ds=\frac{K_{HE}}{1000}*8.72*4200*(H(\overline{V})-dh(4200)-5.1)\\
K_H\int_0^T\phi_T(s)ds=K_H*4200*3600
\end{cases}
\end{equation*}
from where we deduce
\begin{equation*}
K_H(K_{HE})=\frac{K_{HE}*8.72*(H(\overline{V})-dh(4200)-5.1)}{3.6*10^6}.
\end{equation*}
Then $K_{HE}=0.3\implies K_H=1.8*10^{-5}$.

\subsection{Minimum cost condition}

Given the nonlinearity in the cost of the turbine flow, we will find the optimal distribution of power. We will omit the time dependence for conformity.
\begin{multline*}
P_H=\\
=\eta\phi_T(H(V)-dh(\phi_T)-5.1)\\
=\eta\phi_T(H(V)-1.27*10^{-3}\phi_T-5.1)\\
=\underbrace{[\eta(H(V)-5.1)]}_{=K_1}\phi_T+\underbrace{[-\eta1.27*10^{-3}]}_{=K_2\in\R^-}\phi_T^2\\
=K_1\phi_T+K_2\phi_T^2.
\end{multline*}
Now we define $\overline{K}_2=-K_2\in\R^+$ and try to find $\phi_T(P_H)$, then
\begin{multline*}
P_H=-\overline{K}_2\phi_T^2+K_1\phi_T\implies0=-\overline{K}_2\phi_T^2+K_1\phi_T-P_H\implies\\
\phi_T=\frac{-K_1\pm\sqrt{K_1^2-4(-\overline{K}_2)(-P_H)}}{2(-\overline{K}_2)}\implies
\phi_T=\frac{K_1-\sqrt{K_1^2-4\overline{K}_2P_H}}{2\overline{K}_2}.
\end{multline*}
For the last implication, we use that for zero power we have zero flow. Now we have $D=P_F+P_H=\phi_F+P_H$ and the cost of the Hamiltonian is $C=\hat{K}_F\phi_F+\hat{K}_H\phi_T$, then writing $P_H=D-\phi_F$ and using our expression for $\phi_T(P_H)$ we have
\begin{equation*}
C(P_H)=\hat{K}_F(D-\phi_H)+\frac{\hat{K}_H}{2\overline{K}_2}\Bigg(K_1-\sqrt{K_1^2-4\overline{K}_2P_H}\Bigg)
\end{equation*}
and
\begin{equation*}
\frac{\partial C}{\partial P_H}(P_H)=\frac{\hat{K}_H}{\sqrt{K_1^2-4\overline{K}_2P_H}}-\hat{K}_F.
\end{equation*}
Then we want to know until which point $C(P_H)$ is a decreasing function then we want $P_H^*$ such that $\frac{\partial C}{\partial P_H}(P_H^*)=0$, this is
\begin{equation*}
P_H^*=\frac{1}{4\overline{K}_2}\Bigg[K_1^2-\Bigg(\frac{\hat{K}_H}{\hat{K}_F}\Bigg)^2\Bigg].
\end{equation*}
Then the distribution of power is given by
\begin{equation*}
\begin{cases}
P_H=D,P_F=0\ &\text{if}\ D\leq P_H^*,D\leq\overline{P}_H\\
P_H=\overline{P}_H,P_F=D-\overline{P}_H\ &\text{if}\ D\leq P_H^*,D>\overline{P}_H\\
P_H=P_H^*,P_F=D-P_H^*\ &\text{if}\ D>P_H^*,D\leq\overline{P}_H\\
P_H=P_H^*,P_F=D-P_H^*\ &\text{if}\ D>P_H^*,D>\overline{P}_H,P_H^*\leq\overline{P}_H\\
P_H=\overline{P}_H,P_F=D-\overline{P}_H\ &\text{if}\ D>P_H^*,D>\overline{P}_H,P_H^*>\overline{P}_H
\end{cases}
\end{equation*}
where $\overline{P}_H=\eta*4200*(H(V)-dh(4200)-5.1)$.

\end{document}