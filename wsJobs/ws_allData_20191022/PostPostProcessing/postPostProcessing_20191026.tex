%%%%%%%%%%%%%%%%%%%%%%%%%%%%%%%%%%%%%%%%%
% Beamer Presentation
% LaTeX Template
% Version 1.0 (10/11/12)
%
% This template has been downloaded from:
% http://www.LaTeXTemplates.com
%
% License:
% CC BY-NC-SA 3.0 (http://creativecommons.org/licenses/by-nc-sa/3.0/)
%
%%%%%%%%%%%%%%%%%%%%%%%%%%%%%%%%%%%%%%%%%

%------------------------------------------------
%	PACKAGES AND THEMES
%------------------------------------------------

\documentclass[aspectratio=169]{beamer}

\mode<presentation> {

% The Beamer class comes with a number of default slide themes
% which change the colors and layouts of slides. Below this is a list
% of all the themes, uncomment each in turn to see what they look like.

%\usetheme{default}
%\usetheme{AnnArbor}
%\usetheme{Antibes}
%\usetheme{Bergen}
%\usetheme{Berkeley}
%\usetheme{Berlin}
%\usetheme{Boadilla}
%\usetheme{CambridgeUS}
%\usetheme{Copenhagen}
%\usetheme{Darmstadt}
%\usetheme{Dresden}
%\usetheme{Frankfurt}
%\usetheme{Goettingen}
%\usetheme{Hannover}
%\usetheme{Ilmenau}
%\usetheme{JuanLesPins}
%\usetheme{Luebeck}
\usetheme{Madrid}
%\usetheme{Malmoe}
%\usetheme{Marburg}
%\usetheme{Montpellier}
%\usetheme{PaloAlto}
%\usetheme{Pittsburgh}
%\usetheme{Rochester}
%\usetheme{Singapore}
%\usetheme{Szeged}
%\usetheme{Warsaw}

% As well as themes, the Beamer class has a number of color themes
% for any slide theme. Uncomment each of these in turn to see how it
% changes the colors of your current slide theme.

%\usecolortheme{albatross}
%\usecolortheme{beaver}
%\usecolortheme{beetle}
%\usecolortheme{crane}
%\usecolortheme{dolphin}
%\usecolortheme{dove}
%\usecolortheme{fly}
%\usecolortheme{lily}
%\usecolortheme{orchid}
%\usecolortheme{rose}
%\usecolortheme{seagull}
%\usecolortheme{seahorse}
%\usecolortheme{whale}
%\usecolortheme{wolverine}

%\setbeamertemplate{footline} % To remove the footer line in all slides uncomment this line
%\setbeamertemplate{footline}[page number] % To replace the footer line in all slides with a simple slide count uncomment this line

\setbeamertemplate{navigation symbols}{} % To remove the navigation symbols from the bottom of all slides uncomment this line
}

\usepackage{graphicx} % Allows including images
\usepackage{booktabs} % Allows the use of \toprule, \midrule and \bottomrule in tables
\usepackage{subfig}
\usepackage{bm}
\usepackage{array}
\usepackage{commath}
\usepackage{makecell}
\usepackage{xcolor}
\usepackage{graphicx}
\usepackage{amsmath}
\usepackage{wasysym}
\usepackage{mathtools}
\usepackage[linesnumbered,ruled,vlined]{algorithm2e}
\newcommand{\E}{\mathbb{E}}
\newcommand{\R}{\mathbb{R}}

%------------------------------------------------
%	TITLE PAGE
%------------------------------------------------

\title[SOC of Renewable Energy]{Stochastic Optimal Control of Renewable Energy:\\
Post Post Processing} % The short title appears at the bottom of every slide, the full title is only on the title page

\author{Renzo Caballero} % Your name
\institute[KAUST] % Your institution as it will appear on the bottom of every slide, may be shorthand to save space
{renzo.caballerorosas@kaust.edu.sa\\
\quad\\
\medskip
Advisor: Professor Ra\'ul Tempone\\
}

\begin{document}

\begin{frame}
\titlepage % Print the title page as the first slide
\end{frame}

%------------------------------------------------
%	PRESENTATION SLIDES
%------------------------------------------------

\begin{frame}
\frametitle{Definitions}

We do an abuse of notaton and we call:

\begin{enumerate}

\item \alert{Value function} to the value of the HJB solution at the initial point.

\item \alert{Running cost} to the integration of the running cost over the simulated optimal path and simulated optimal controls.

\item \alert{Historical running cost} to the integration of the running cost over the historical path and historical controls.

\end{enumerate}

\end{frame}


\begin{frame}
\frametitle{Discretization relative error over almost 500 days}
\begin{columns}[c]

\column{.5\textwidth}
\includegraphics[width=1\columnwidth]{Plots/Plot_1.eps}

\column{.4\textwidth}
I still have to check a bit more carefully, but until now, the complicated days are:
\begin{enumerate}

\item Days where HJB does not use fuel, but the OP does (a bit).
\item Days where the cost is extremely low, then a small absolute error relatively large.

\end{enumerate}

\end{columns}

\end{frame}


\begin{frame}
\frametitle{Error and small water value}
\begin{columns}[c]

\column{.4\textwidth}
\includegraphics[width=1\columnwidth]{Plots/Plot_1.eps}

\column{.4\textwidth}
\includegraphics[width=1\columnwidth]{Plots/Plot_4.eps}

\end{columns}
\quad\\
We can see as the error increases when the water has minimal value. This effect motivates the idea of an absolute (and not only relative) error criteria.

\end{frame}


\begin{frame}
\frametitle{Error and small cost}
\begin{columns}[c]

\column{.4\textwidth}
\includegraphics[width=1\columnwidth]{Plots/Plot_1.eps}

\column{.4\textwidth}
\includegraphics[width=1\columnwidth]{Plots/Plot_6.eps}

\end{columns}
\quad\\
One more time, we see that the relative error is significant when the value function is minimal. A small value function is associated with water with zero value.
\end{frame}


\begin{frame}
\frametitle{Absolute and relarive error}
\begin{columns}[c]

\column{.4\textwidth}
\includegraphics[width=1\columnwidth]{Plots/Plot_10.eps}

\column{.4\textwidth}
\includegraphics[width=1\columnwidth]{Plots/Plot_11.eps}

\end{columns}
\quad\\
\alert{The plots above only contemplate red days}. We can observe that the absolute error is usually smaller than 0.2 HT-USD. The large absolute error is associated with the use of fuel during the optimal path.

\end{frame}


\begin{frame}
\frametitle{Simulated and historical cost}
\begin{columns}[c]

\column{.4\textwidth}
\includegraphics[width=1\columnwidth]{Plots/Plot_6.eps}

\column{.4\textwidth}
\includegraphics[width=1\columnwidth]{Plots/Plot_9.eps}

\end{columns}
\quad\\
We can see how the historical cost is higher than the simulated one. However, one main reason is because of the virtual control, which has a lot of influence on the cost when $\lambda=0$, and the value has non-zero cost.

\end{frame}

\end{document} 