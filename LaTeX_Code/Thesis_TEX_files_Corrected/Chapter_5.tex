\chapter{Results}

In this chapter, we will show some of the most relevant results of the Subproblems presented in previous chapters.

\section{One Dam and One FFS} \label{First_Subproblem_Results}

In this Section we present the results of the First Subproblem formulated mathematical and numerically in Subsections \ref{First_Subproblem} and \ref{First_Subproblem_Num} respectively.\\

As this Subproblem has one single spatial dimension, we can show all the figures evolving in time.\\

In Fig. (\ref{Plot_SP1_1})-(a) we can see the value function evolving in time. It accumulates more cost when the state of the dam is below $\underline{v}^{(1)}$ (recall $\underline{v}^{(1)}=\SI{3.5e3}{hm^3}$). Under the condition $v^{(1)}\leq\underline{v}^{(1)}$ the dam is not able to produce power and all the demand must be supplied from the FFS, which is more expensive.\\
In Fig. (\ref{Plot_SP1_1})-(b) we can see the value function's derivative w.r.t. the state $v^{(1)}$. The effect of a small change in $v^{(1)}$ becomes relevant when $v^{(1)}\approx\underline{v}^{(1)}$ because it is the limit that separates the cases where we use the dam at its maximum capacity, or we turn it off totally.

\begin{figure}[H]
\centering
\subfloat[Value function for different $t_0$ and $v^{(1)}_0$. We can see how it is accumulating the running cost as it evolves in time.]{\includegraphics[width=0.45\columnwidth]{Figures/Constant_Demand_1.eps}}\qquad
\subfloat[Partial derivative w.r.t. $v^{(1)}_0$ of the value function.]{\includegraphics[width=0.45\columnwidth]{Figures/Constant_Demand_2.eps}}
\caption{We can see in (a) the value function and in (b) its partial derivative w.r.t. $v^{(1)}_0$. The scale makes the function in (a) look discontinuous. However, as it is an accumulated cost over time it is continuous.}
\label{Plot_SP1_1}
\end{figure}

In Fig. (\ref{Plot_SP1_2}) we can see the direct effect of the dam's behavior over the Hamiltonian. When the dam is not producing power, and the FFS supplies all the demand, the instant cost is higher.\\
It is also possible in Fig. (\ref{Plot_SP1_2})-(a) to notice how the instant cost increases when the dam has less water. This effect is due to the reduction in efficiency.

\begin{figure}[H]
\centering
\subfloat[Value of the Hamiltonian as a function of $t$ and $v^{(1)}$. We can see how the cost is higher when the reservoir is empty.]{\includegraphics[width=0.45\columnwidth]{Figures/Constant_Demand_3.eps}}\qquad
\subfloat[Turbined flow as a function of $t$ and $v^{(1)}$. We can see how for $v^{(1)}\leq\underline{v}^{(1)}$ the flow is exactly zero.]{\includegraphics[width=0.45\columnwidth]{Figures/Constant_Demand_4.eps}}
\caption{We can see the value of the Hamiltonian in (a) and the turbined flow in (b). When the volume in under its limit, the turbined flow becomes zero and all the demand must be satisfies from fuel, which is more expensive than the water. This effect is represented in the value of the Hamiltonian.}
\label{Plot_SP1_2}
\end{figure}

In Fig. (\ref{Plot_SP1_3})-(a) we can see the FFS's control. As it is expected, it increases dramatically then $v^{(1)}\leq\underline{v}^{(1)}$. In Fig. (\ref{Plot_SP1_3})-(b) it is represented the demand over time, which is constant. We use a constant demand to simplify the behavior of the plots.

\begin{figure}[H]
\centering
\subfloat[Fossil fuel control, we can see hot it is equal to the demand when $v^{(1)}\leq\underline{v}^{(1)}$. When we compare with Fig. (\ref{Plot_SP1_2})-(a) it becomes clear how the cost of the Hamiltonian is governed by the fuel.]{\includegraphics[width=0.45\columnwidth]{Figures/Constant_Demand_5.eps}}\qquad
\subfloat[Demand as a function of the time. We choose a constant as it is the First Subproblem.]{\includegraphics[width=0.45\columnwidth]{Figures/Constant_Demand_6.eps}}
\caption{We can see the fuel control in (a) and the used demand in (b). We can see how the value of the Hamiltonian is dominated by fuel.}
\label{Plot_SP1_3}
\end{figure}

In Fig. (\ref{Plot_SP1_4}) we repeat the plots of Fig. (\ref{Plot_SP1_1}), but performing the simulation in an interval closer to $\underline{v}^{(1)}$.

\begin{figure}[H]
\centering
\subfloat[Value function for different $t_0$ and $v^{(1)}_0$. We can see how it is accumulating the running cost as it evolves in time.]{\includegraphics[width=0.45\columnwidth]{Figures/Constant_Demand_Compressed_1.eps}}\qquad
\subfloat[Partial derivative w.r.t. $v^{(1)}_0$ of the value function.]{\includegraphics[width=0.45\columnwidth]{Figures/Constant_Demand_Compressed_2.eps}}\\
\caption{We can see in (a) the value function and in (b) its partial derivative w.r.t. $v^{(1)}_0$. We can see how when we zoom in the scale, the value function is continuous.}
\label{Plot_SP1_4}
\end{figure}

In Fig. (\ref{Plot_SP1_5}) we perform the computation of the value function over all the domain and over the space where the trajectory can be (see Subsection \ref{Subsection_Transformation_Dams}). We overlap the two results to show the matching between the solutions.

\begin{figure}[H]
\centering
\includegraphics[width=0.6\columnwidth]{Figures/Comparison.eps}
\caption{The numerical solution of the HJB equation with and without the change of variables explained in Subsection \ref{Subsection_Transformation_Dams}. As we can see, in the region where the PDEs share domain, they obtain the same value.}
\label{Plot_SP1_5}
\end{figure}

\section{Two Dams and One FFS} \label{Second_Subproblem_Results}

In this Section we present the results of the Second Subproblem formulated mathematical and numerically in Subsections \ref{Second_Subproblem} and \ref{Second_Subproblem_Num} respectively.\\

All the surface plots in this Subsection are at time $t_0$. Recall that the second dam with state $\hat{v}^{(2)}$ can produce more than five times the power of the first dam.\\

In Fig. (\ref{Plot_SP2_1})-(a) we can see the value function at time $t_0$. As expected, it has a stronger dependence w.r.t. the second dam.\\
In Fig. (\ref{Plot_SP2_1})-(b) we can see the demand used in this Subproblem, which was presented in Eq. (\ref{Demand_SSP}).

\begin{figure}[H]
\centering
\subfloat[Value function at time $t_0$.]{\includegraphics[width=0.45\columnwidth]{Figures/2Dams_1.eps}}\qquad
\subfloat[Demand over time. The equation used is Eq. (\ref{Demand_SSP}).]{\includegraphics[width=0.45\columnwidth]{Figures/2Dams_8.eps}}
\caption{Value function (a) and demand (b). We can see the strong dependence between the value function and the state of the second dam.}
\label{Plot_SP2_1}
\end{figure}

In Fig. (\ref{Plot_SP2_2}) we can see the instant hydropower of both dams at time $t_0$. The first dam is working at its maximum capacity for all the state-space, we can see the loss of efficiency as the dam has less water in its reservoir.\\
The second dam is working at its maximum capacity for $\hat{v}^{(2)}\lesssim0.9$ and when the volume is larger, the system is able to satisfy all the demand from hydropower.

\begin{figure}[H]
\centering
\subfloat[Instant power of the first dam at time $t_0$.]{\includegraphics[width=0.45\columnwidth]{Figures/2Dams_3.eps}}\qquad
\subfloat[Instant power of the second dam at time $t_0$.]{\includegraphics[width=0.45\columnwidth]{Figures/2Dams_4.eps}}
\caption{Hydropower of the first dam (a) and the second dam (b) as a function of the states at time $t_0$.}
\label{Plot_SP2_2}
\end{figure}

In Fig. (\ref{Plot_SP2_3}) we can see the partial derivatives of the value function w.r.t. the states at time $t_0$. In both cases, the magnitude is larger when the value of the state is smaller, this implies that the less water in the reservoirs, the more valuable it is.

\begin{figure}[H]
\centering
\subfloat[Derivative of the value function w.r.t. $\hat{v}^{(1)}$ at time $t_0$.]{\includegraphics[width=0.45\columnwidth]{Figures/2Dams_5.eps}}\qquad
\subfloat[Derivative of the value function w.r.t. $\hat{v}^{(2)}$ at time $t_0$.]{\includegraphics[width=0.45\columnwidth]{Figures/2Dams_6.eps}}
\caption{Partial derivatives of the value function w.r.t. the states at time $t_0$. The derivative w.r.t. the second dam is greater in magnitude because this dam is more relevant in the system (i.e., it is significantly bigger than the first dam).}
\label{Plot_SP2_3}
\end{figure}

In Fig. (\ref{Plot_SP2_4})-(a) we can see the Hamiltonian at time $t_0$. Even when most of the power is hydropower, the small amount of fuel used at time $t_0$ is expensive enough to predominate in the shape of the Hamiltonian (compare (a) and (b)).

\begin{figure}[H]
\centering
\subfloat[Hamiltonian at time $t_0$.]{\includegraphics[width=0.45\columnwidth]{Figures/2Dams_7.eps}}\qquad
\subfloat[FFS's instant power at time $t_0$.]{\includegraphics[width=0.45\columnwidth]{Figures/2Dams_2.eps}}
\caption{Hamiltonian (a) and FFS's instant power (b) at time $t_0$. We can see the strong dependency between the use of fuel and the value of the Hamiltonian.}
\label{Plot_SP2_4}
\end{figure}

\section{Stochastic Wind Power} \label{Third_Subproblem_Results}

In this Section we present the results of the Third Subproblem formulated mathematical and numerically in Subsections \ref{Third_Subproblem} and \ref{Third_Subproblem_Num} respectively.\\

This is the first Subproblem where the concept of Effective Demand (defines in Subsection \ref{Subsection_ED}) is used. All the surface plots in this Subsection are at time $t_0$. Recall that the second dam with state $\hat{v}^{(2)}$ can produce more than five times the power of the first dam. The change of variable in the wind power direction maps $F_{\hat{w}}:[0,1]\to[0.3,0.7]$ as we can see in Fig. (\ref{Plot_SP3_5}).\\

In Fig. (\ref{Plot_SP3_1}) we can see a confidence interval for the wind power SDE (\ref{Wind_SDE}) computed matching the analytic moments of the SDE with a beta distribution at each time $t\in[t_0,t_1]$. As we can see, the change of variable described in Subsection \ref{Subsection_Transformation_Wind}, \textit{which confines the domain of the PDE (\ref{HJB_Subproblem_3})  in the wind direction to the area covered by the confidence interval}, saves computational work as we are solving the equation in a reasonably smaller interval.\\

\begin{figure}[H]
\centering
\includegraphics[width=0.45\columnwidth]{Figures/Wind_Space.eps}
\caption{The confidence interval for the wind power SDE (\ref{Wind_SDE}). It is computed the Euler-Maruyama method described in Subsection \ref{Subsection_Transformation_Wind}.}
\label{Plot_SP3_1}
\end{figure}

In Fig. (\ref{Plot_SP3_2})-(a) we can see the Effective Demand over time (recall definition in Subsection \ref{Subsection_ED}), where the demand is represented by Eq. (\ref{Demand_3SP}). The red vertical lines represent the time discretization. In Fig. (\ref{Plot_SP3_2})-(b) we can see the value function at time $t_0$. There is a surface for each one of the wind power discretizations, naturally, the more wind power we have (which also implies the less Effective Demand), the less the value function is.

\begin{figure}[H]
\centering
\subfloat[Effective Demand (see Subsection \ref{Subsection_ED}) over time. We are using 9 discretizations in the direction of the wind power.]{\includegraphics[width=0.4\columnwidth]{Figures/DamsAndWind_4.eps}}\qquad
\subfloat[Value function for each one of the wind power values at time $t_0$. Naturally, when we have more wind, the value function has a smaller value.]{\includegraphics[width=0.4\columnwidth]{Figures/DamsAndWind_5.eps}}
\caption{We solve the HJB equation using FD. In (a) we can see the equivalent to the discretizations in the direction of the wind power, which is the Effective Demand (see Subsection \ref{Subsection_ED}). In (b) the value function at time $t_0$ for each wind power discretization.}
\label{Plot_SP3_2}
\end{figure}

In Fig. (\ref{Plot_SP3_3}) we can see the partial derivatives of the value function at time $t_0$ w.r.t. the volume in the dams. As is expected, both derivatives are always negative, which shows the reduction in the value function when we have more water. Again, we can see as a difference in the volume has more effect when the dams are near their low limit, and as the second dam is more relevant in the total cost of the system.

\begin{figure}[H]
\centering
\subfloat[Derivative of the cost function at final time w.r.t. the volume $\hat{v}^{(1)}$. We can see how is always negative, which implies that to have more water, reduces the cost.]{\includegraphics[width=0.4\columnwidth]{Figures/DamsAndWind_6.eps}}\qquad
\subfloat[Derivative of the cost function at final time w.r.t. the volume $\hat{v}^{(2)}$. We can see how is always negative, which implies that to have more water, reduces the cost.]{\includegraphics[width=0.4\columnwidth]{Figures/DamsAndWind_7.eps}}
\caption{We can see the derivatives of the cost function w.r.t. the volumes of water. As the second dam is more relevant in the system (it generates more power), the value of the derivative w.r.t. this dam is greater in magnitude.}
\label{Plot_SP3_3}
\end{figure}

In Fig. (\ref{Plot_SP3_4}) we can see the first and second derivative of the value function at time $t_0$ w.r.t. wind power. As expected, the first derivative is always negative, implying that the more wind, the less total cost. The second derivative is a bit more tricky: It is always positive, implying that even when to have more wind power reduces the total cost, the reduction is getting smaller with more wind power.

\begin{figure}[H]
\centering
\subfloat[Derivative of the value function at time $t_0$ w.r.t. wind power $\hat{w}$. We can see that it is always negative, which implies that to have more wind power, reduces the total cost.]{\includegraphics[width=0.4\columnwidth]{Figures/DamsAndWind_8.eps}}\qquad
\subfloat[Second derivative of the value function at time $t_0$ w.r.t. wind power $\hat{w}$. We can see that it is always positive, which implies that to have more wind power reduces the cost, but each time in a smaller ratio.]{\includegraphics[width=0.4\columnwidth]{Figures/DamsAndWind_9.eps}}
\caption{First and second derivative of the value function at time $t_0$ w.r.t. wind power $\hat{w}$.}
\label{Plot_SP3_4}
\end{figure}

In Fig. (\ref{Plot_SP3_5}) we can see the value function and its partial derivative w.r.t. the wind power $\hat{w}$. We can see that always to have more wind power, reduces the accumulated cost. This idea is related with the fact that, the more wind, we achieve a smaller Effective Demand.

\begin{figure}[H]
\centering
\subfloat[The value function as a function of the wind power $\hat{w}$ for each time steps.]{\includegraphics[width=0.45\columnwidth]{Figures/DamsAndWind_1.eps}}\quad
\subfloat[All the partial derivatives w.r.t. the wind power $\hat{w}$ for each time step. We can see that it is always negative, but increasing.]{\includegraphics[width=0.45\columnwidth]{Figures/DamsAndWind_2.eps}}
\caption{Value function and its partial derivative w.r.t. the wind power $\hat{w}$.}
\label{Plot_SP3_5}
\end{figure}

\section{Simple Linked System} \label{Fourth_Subproblem_Results}

In this Section we present the results of the Fourth Subproblem formulated mathematical and numerically in Subsections \ref{Fourth_Subproblem} and \ref{Fourth_Subproblem_Num} respectively.\\

This is the first Subproblem where we use the continuous-time Lagrangian Relaxation introduces in Subsection \ref{CTLR}.\\

In Fig. (\ref{Plot_SP4_1}) we can compare the value function at time $t_0$ for a system where both dams are connected (Linked), and a system with independent dams (No Linked). As we described in Subsection \ref{Fourth_Subproblem_Num}, we are considering two cases, each case corresponds to a subfigure.

\begin{enumerate}

\item[$\bullet$]First case (subfigure (a)): We can see the value function for the no linked system (orange) and the value of the dual function for the linked system (blue). Naturally, when we connect the dams, the total cost get reduces as we are recycling the water used by the first dam, also we are adding an extra control in the dual problem. We know that the dual function is a lower bound for the primal problem (recall Subsection \ref{Properties_DF}), then is reasonable the reduction is the accumulated cost that we can see compared with the no linked system.\\
Other notable effect is the non-smoothness of the dual function, even it is non-smooth in the optimal Lagrange Multiplayer when we use $\lambda(t)=k\in\R$ constant for $t\in[t_0,t_1]$.

\item[$\bullet$] Second case (subfigure (b)): In this case, the second (downstream) dam as its efficiency fixed. Then, when $\lambda(t)\equiv0$, the system does not receive any benefice from the virtual control, and the cost associated with the first dam also is not affected. For this reason, we can observe that the accumulated cost of the linked (blue) and no linked (orange) systems coincides in $\lambda(t)\equiv0$.\\
Again we can notice the non-smoothness but concavity of the dual function.

\end{enumerate}

\begin{figure}[H]
\centering
\subfloat[Case when the efficiency of both dams depends on their water levels. This figure puts in evidence the non-smoothness of the dual function.]{\includegraphics[width=0.45\columnwidth]{Figures/Test_Lambda_1.eps}}\quad
\subfloat[Case when only the efficiency of the first dam depends on its water level, and the efficiency of the second dam is fixed.]{\includegraphics[width=0.45\columnwidth]{Figures/Test_Lambda_2.eps}}
\caption{Comparison between the value function for a system with independent dams, and one connected but relaxed. The plot is as a function of a constant-in-time $\lambda(t)$.}
\label{Plot_SP4_1}
\end{figure}

\section{All Dams, a FFS and a Battery} \label{Fifth_Subproblem_Results}

In this Section we present the results of the Fifth Subproblem formulated mathematical and numerically in Subsections \ref{Fifth_Subproblem} and \ref{Fifth_Subproblem_Num} respectively.\\

In Fig. (\ref{Plot_SP5_1}) we can see the results of the convergence tests of this Subproblem. More convergence tests were made for all Subproblems. However, the inclusion of all of them would be tedious.\\
As we can see, the relative error is decreasing as we increase the number of discretizations. We reach a relative error of order $\SI{e-3}{}$, which is enough considering that we are estimating a total cost.

\begin{figure}[H]
\centering
\subfloat[Both convergence test, system with battery is connected and disconnected. We used as amount of partitions $2^{i}$, $2^{i-1}$ $2^{i+4}$ for the dams, battery and time respectively.]{\includegraphics[width=0.45\columnwidth]{Figures/F1.eps}}\\
\subfloat[Relative error for the convergence test for the system using the battery.]{\includegraphics[width=0.45\columnwidth]{Figures/F2.eps}}\quad
\subfloat[Relative error for the convergence test with the battery disconnected.]{\includegraphics[width=0.45\columnwidth]{Figures/F3.eps}}
\caption{Convergence tests. We realize convergence tests in a system with and without battery. We plot the relative error with respect to the last point of the simulation.}
\label{Plot_SP5_1}
\end{figure}

\section{Complete Linked System (no Battery)} \label{Sixth_Subproblem_Results}

In this Section we present the results of the Sixth Subproblem formulated mathematical and numerically in Subsections \ref{Sixth_Subproblem} and \ref{Sixth_Subproblem_Num} respectively.\\

In Fig. (\ref{Plot_SP6_1}) we can see the minimizer of the dual problem, and we can compare it with the use of fuel.\\
There exists a strong relationship between the value of the Lagrangian Multiplier $\hat{\lambda}_{21}(t)$ and the use of fuel in the system. During the period when we are using fuel, given that its cost is higher than the water's one, we want to have all the hydropower at its maximum capacity to reduce the running cost. However, the ROR dam (recall definition in Subsection \ref{ROR_Dam}) it not able to work at its maximum unless $\tau_{21}$ time before, its upstream dam uses enough amount of water.\\
Looking at the way  we construct the relaxation (\ref{Dual_Function_Sixth}), we notice that if for $\overline{t}\in[\tau_{21},t_1]$ we have a large value in $\hat{\lambda}_{21}(\overline{t})$, we are motivating the first dam to use its water at time $\overline{t}-\tau_{21}$ due to a reduction in its costs at that time. Then the ROR dam can generate at full capacity at time $\overline{t}$, reducing the total accumulated cost.\\
The effect of $\hat{\lambda}_{32}$ is smaller as it is only related with the efficiency of the third dam (see the Fourth Subproblem in Subsection \ref{Fourth_Subproblem_Results}), and not with instant power production as $\hat{\lambda}_{21}$.

\begin{figure}[H]
\centering
\subfloat[Value at the final iteration of the Lagrangian Multiplier $\hat{\lambda}_{21}$.]{\includegraphics[width=0.45\columnwidth]{Figures/OP_2.eps}}\qquad
\subfloat[Controls of the FFSs over time.]{\includegraphics[width=0.4\columnwidth]{Figures/OP_7.eps}}
\caption{We can see the results of the minimization of the dual function and compare it with the use of fuel in the system. In (a) the optimal Lagrange Multiplier an in (b) the controls of the FFSs over time.}
\label{Plot_SP6_1}
\end{figure}

In Fig. (\ref{Plot_SP6_2}) we can see the outputs of our Oracle (defined in Subsection \ref{Subsection_Oracle}). In subfigure (a) we can see the number of needed iterations before satisfying the convergence criteria, also the evaluations of the dual function by the Oracle.\\
In subfigure (b) we can see the last used subgradient before convergence. We can see that the larger component in (b) is given at the same time that we use fuel.

\begin{figure}[H]
\centering
\subfloat[Final value of the Lagrangian Multiplier $\hat{\lambda}_{21}$.]{\includegraphics[width=0.45\columnwidth]{Figures/OP_3.eps}}\qquad
\subfloat[Subgradient used at the last iteration.]{\includegraphics[width=0.45\columnwidth]{Figures/OP_4.eps}}
\caption{Results of the minimization of the dual function. In (a) the evaluations of the Oracle and in (b) the subgradient used at the last iteration.}
\label{Plot_SP6_2}
\end{figure}

\section{Complete Linked System with Battery} \label{Seventh_Subproblem_Results}

In this Section we present the results of the Seventh Subproblem formulated mathematical and numerically in Subsections \ref{Seventh_Subproblem} and \ref{Seventh_Subproblem_Num} respectively.\\

\subsubsection{Time Window of One Day}

With the help of the KAUST's cluster Ibex, we are able to simulate 59 days in a reasonable time in only some hours (using nodes with 40 cores each one).\\

In Fig. (\ref{Plot_SP7_0}) we can see the power balance along the optimal path for a given day. In this plot we include all the generators, the demand and the exports.

\begin{figure}[H]
\centering
\includegraphics[width=0.8\columnwidth]{Figures/Simulation_Day_16/101.eps}
\caption{Power balance over one day. The battery gets charged when cheaper generators have slack capacity, using later all its energy to replace the more expensive generators.}
\label{Plot_SP7_0}
\end{figure}

In Fig. (\ref{Plot_SP7_1}) we can see an analogue to Fig. (\ref{Plot_SP7_0}), where we only show the controllable sources. 

\begin{figure}[H]
\centering
\includegraphics[width=0.8\columnwidth]{Figures/Simulation_Day_16/120.eps}
\caption{Similar to Fig. (\ref{Plot_SP7_0}) but showing only the controllable sources of power.}
\label{Plot_SP7_1}
\end{figure}

In Fig. (\ref{Plot_SP7_2}) we can see the state of the battery over time and the optimal path. Notice as the battery is getting charged while we are using the cheaper FFS, to discharge when we use the most expensive one.

\begin{figure}[H]
\centering
\includegraphics[width=0.4\columnwidth]{Figures/Simulation_Day_16/Extra_2049.eps}
\caption{State (capacity) of the battery over time.}
\label{Plot_SP7_2}
\end{figure}

In Fig. (\ref{Plot_SP7_3}) we can see the controls of the generators over the optimal path. As the demand is high and we have to use the FFSs, all the dams are working at their maximum capacity all the time.\\
We can notice that Baygorria is using spillage, it is because, on this particular day, that dam had an overflow of water.

\begin{figure}[H]
\centering
\subfloat[Controls of the dams over time. In this particular day, Baygorria is having an overload of water and needs to use spillage.]{\includegraphics[width=0.4\columnwidth]{Figures/Simulation_Day_16/103.eps}}\qquad
\subfloat[Controls of the FFSs over time.]{\includegraphics[width=0.4\columnwidth]{Figures/Simulation_Day_16/104.eps}}
\caption{Controls of the dams in (a) where we can see that they are working at their full capacity. In (b) the fussil fuel stations (FFSs) controls.}
\label{Plot_SP7_3}
\end{figure}

In Fig. (\ref{Plot_SP7_4}) we can see the total energy and cost distributions. Recall from Subsection \ref{Subsection_NoControllable} that the non-controllable sources do not have associated cost.\\
Even when less than 12 \% of the total energy is from fossil fuel, it represents more than 90 \% of the total cost.

\begin{figure}[H]
\centering
\subfloat[Energy distribution during the day.]{\includegraphics[width=0.4\columnwidth]{Figures/Simulation_Day_16/116.eps}}\qquad
\subfloat[Cost distribution during the day.]{\includegraphics[width=0.4\columnwidth]{Figures/Simulation_Day_16/117.eps}}
\caption{We can see the energy (a) and cost (b) distributions during the day.}
\label{Plot_SP7_4}
\end{figure}

In Fig. (\ref{Plot_SP7_5}) we can see in the same plot, the value function (solution of the HJB equation at $\bm{x}=\bm{x}_0$) evaluated at the initial state (in orange), and the accumulated cost over the optimal path (in blue). If we compare with Fig. (\ref{Plot_SP7_1}), we can see the accumulated cost increasing faster when we use more amount of fuel.

\begin{figure}[H]
\centering
\includegraphics[width=0.4\columnwidth]{Figures/Simulation_Day_16/102.eps}
\caption{Value function at the initial state (orange) and accumulated cost over the optimal path (blue). See Subsection \ref{Subsection_OP} for optimal path definition.}
\label{Plot_SP7_5}
\end{figure}

\subsubsection{Time Window of One Week}

To test the software, we repeat the experiment but using at time windows of a week. Figures (\ref{Plot_SP7_6}) and (\ref{Plot_SP7_7}) are the analog to Figures (\ref{Plot_SP7_0}) and (\ref{Plot_SP7_1}).\\

In Fig. (\ref{Plot_SP7_6}) we can see the power balance along the optimal path for a given week. In this plot we include all the generators, the demand and the exports.\\
This plot put in evidence the high variability of the solar and wind power, and in the exports (compare demand with demand + exports).

\begin{figure}[H]
\centering
\includegraphics[width=1\columnwidth]{Figures/1W_0.eps}
\caption{Power balance over a week.}
\label{Plot_SP7_6}
\end{figure}

In Fig. (\ref{Plot_SP7_7}) we can see an analogue to Fig. (\ref{Plot_SP7_6}), where we only show the controllable sources. 

\begin{figure}[H]
\centering
\includegraphics[width=1\columnwidth]{Figures/1W_1.eps}
\caption{Similar to Fig. (\ref{Plot_SP7_6}) but showing only the controllable sources of electric energy. We can see how the battery always is replacing fuel when active.}
\label{Plot_SP7_7}
\end{figure}

%In Fig. (\ref{Plot_SP7_8}) we can see the state of the battery over time and the optimal path.
%
%\begin{figure}[H]
%\centering
%\includegraphics[width=0.5\columnwidth]{Figures/Battery_1W.eps}
%\caption{State (capacity) of the battery over a week.}
%\label{Plot_SP7_8}
%\end{figure}