\chapter{Introduction}

Uruguay is a pioneer in the use of renewable energy, to the point 
of being able to supply 100\% of its demand from renewable sources. 
The generation facilities use biomass combustion, hydropower, solar and wind 
power \cite{secretariat2018global} as well as fossil fuel driven generators.\\
Wind and solar power generators are mostly not controllable. Also, they exhibit high uncertainty \cite{foley2012current} and variability in relatively short time scales, which span just a few hours. This fact makes stochastic effects relevant when planning the power generation in one day.  In this context, classic scenario-based approaches like \cite{de2010optimal}  become less attractive, as they do not provide optimal policies for all states. In this thesis,  we use stochastic optimal control to construct space-time continuous feasible policies. These policies produce an electric energy dispatch which minimizes the total production cost while satisfying the demand. 
\begin{figure}[ht!]
\centering
\includegraphics[width=0.5\textwidth]{Figures/Flow.pdf}
\caption{Workflow of optimal control for real systems.}
\label{Flow_Control}
\end{figure}\\
Fig. (\ref{Flow_Control}) shows the workflow adopted in this thesis. It involves 
the following steps:

\begin{enumerate}

\item[Step 1]: A physical and engineering description is developed for the real 
system under study. All relevant physical phenomena, limitations, and parameters
 must be identified to understand the behavior of the system. We use official 
 data to model the Uruguayan system provided by the companies ADME \cite{ADME} and UTE \cite{UTE}. The system is shown in Fig. \ref{fig:pptfigure}, where we can see the generators on left, the consumption on the right and a storage device, namely a battery. Throughout this thesis, we disregard the constraints imposed by the power transmission network and model the Uruguayan grid as a single node.
\begin{figure}[ht!]
\centering
\includegraphics[width=0.6\textwidth]{Figures/RENZO.png}
\caption{Representation of the Uruguayan grid with its different kinds of generators on the left and demand sources on the right. On the bottom right, we also include a battery, which is central to this thesis: it can absorb energy during low demand and supply it later during high demand to reduce the use of costly, fuel based generators.}
\label{fig:pptfigure}
\end{figure}

\item[Step 2]: A mathematical model is compiled from the description of the entire 
system, including generators and demand functions. We use physical laws, data fitting, and 
information available for the different components and processes.

\item[Step 3]: Once a mathematical model is in hand, 
we can apply stochastic optimal control. Continuous-time Dynamic Programming (DP)
applies to our model, which includes time-delays,
after a continuous-time Lagrangian Relaxation (LR) to obtain a 
Markovian system. Continuous-time DP is equivalent to solve the Hamilton-Jacobi-Bellman 
(HJB) partial differential equation, whose solution is the value function.

\item[Step 4]: Apply the method and models developed to examples from a real system, in our case, the Uruguayan grid, to assess their performance and have a validation.

\end{enumerate} 
The final step  is to use the optimal controls in the real system and compare
the results with current control strategies. This would require
modifications and tunings with data that is out of scope for this work, but the results
obtained so far indicate that such an effort would be well advised.