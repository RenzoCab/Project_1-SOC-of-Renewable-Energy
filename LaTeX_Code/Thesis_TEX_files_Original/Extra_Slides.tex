%------------------------------------------------

{\setbeamertemplate{footline}{} \setbeamercolor{background canvas}{bg=blue!50}
\begin{frame}[noframenumbering]
\centering
{\Huge Extra.}
\end{frame}}

%------------------------------------------------

{
\setbeamertemplate{footline}{} 
\begin{frame}[noframenumbering]
\frametitle{Subgradient and Subdifferential:}
\begin{columns}[c] % The "c" option specifies centered vertical alignment while the "t" option is used for top vertical alignment

\column{.45\textwidth} % Left column and width
\centering
\includegraphics[width=0.8\columnwidth]{Figures/Subgradient.eps}

\column{.45\textwidth}
\begin{definition} \label{Def_Subgradient}
Let $S\subset\R^m$ be a convex set and $\theta:S\to\R$ be a convex function. 
Then $\bm{\xi}$ is called a \textit{subgradient} of $\theta$ at $\overline{\bm{x}}\in S$ if
\begin{equation*}
\theta(\bm{x})\geq\theta(\overline{\bm{x}})+\bm{\xi}^T(\bm{x}-\overline{\bm{x}})\quad\text{for all}\quad\bm{x}\in S.
\end{equation*}
\end{definition}
\begin{definition} \label{Def_Subdifferential}
The collection $\partial\theta(\overline{\bm{x}})$ of subgradients of $\theta$ in $\overline{\bm{x}}$ is called the subdifferential of $\theta$ in $\overline{\bm{x}}$.
\end{definition}

\end{columns}
\end{frame}
}

%------------------------------------------------

{
\setbeamertemplate{footline}{} 
\begin{frame}[noframenumbering]
\frametitle{Non-smooth minimization stopping condition:}
\begin{columns}[c] % The "c" option specifies centered vertical alignment while the "t" option is used for top vertical alignment

\column{.45\textwidth} % Left column and width
\centering
\includegraphics[width=0.8\columnwidth]{Figures/Subdifferential.eps}

\column{.45\textwidth}
\begin{theorem}
$\bm{x}^*$ is a minimizer of the convex possibly non-smooth function $\theta$ if $\bm{0}\in\partial\theta(\bm{x}^*)$.
\end{theorem}

\end{columns}
\end{frame}
}

%------------------------------------------------

{
\setbeamertemplate{footline}{} 
\begin{frame}[noframenumbering]
\frametitle{Solving the problem forward:}

Once we obtain the optimal $\lambda$ (call it $\lambda^*$), and we have the partial derivatives in all the space, we can compute the optimal path computing the optimal controls at each point:\\
\quad\\
{
\centering{
\begin{minipage}{0.8\textwidth}
\begin{algorithm}[H]
\SetAlgoLined
 initialization: $\bm{x}\leftarrow\bm{x}_0$\;
 \For{$t=1:T-1$}{
  $\phi\leftarrow \arg\min_{\phi\in\bm{\phi}}H(t,\bm{x},\lambda^*,D_{\bm{x}}V)$\tcp*{Minimizing Hamiltonian}
  $\bm{x}\leftarrow \bm{x}+\Delta t\bm{f}(t,\bm{x},\bm{\phi})$\tcp*{Dynamics update}
 }
 \caption{Optimal path computation.}
\end{algorithm}
\end{minipage}}\\
}\quad\\
Where we impose the non-Markovian constraints to guarantee admissible controls.

\end{frame}
}

%------------------------------------------------

{
\setbeamertemplate{footline}{} 
\begin{frame}[noframenumbering]
\frametitle{One week simulated power balance (first week 2019):}
\centering
\includegraphics[width=1\columnwidth]{Figures/1_Week/101.eps}
\end{frame}
}

%------------------------------------------------

{
\setbeamertemplate{footline}{} 
\begin{frame}[noframenumbering]
\frametitle{One week historical power balance (first week 2019):}
\centering
\includegraphics[width=1\columnwidth]{Figures/1_Week/003.eps}
\end{frame}
}

%------------------------------------------------

{
\setbeamertemplate{footline}{} 
\begin{frame}[noframenumbering]
\frametitle{One week cost comparison (first week 2019):}
\begin{figure}[ht!]
\centering
\subfloat[Accumulated cost using the battery.]{\includegraphics[width=0.45\columnwidth]{Figures/1_Week/001.eps}}\quad
\subfloat[Accumulated cost without the battery.]{\includegraphics[width=0.45\columnwidth]{Figures/1_Week/002.eps}}
\end{figure}
\end{frame}
}

%------------------------------------------------

{
\setbeamertemplate{footline}{} 
\begin{frame}[noframenumbering]
\frametitle{One week CFL condition (first week 2019):}
\begin{figure}[ht!]
\centering
\subfloat[CFL condition using the battery.]{\includegraphics[width=0.45\columnwidth]{Figures/1_Week/005.eps}}\quad
\subfloat[CFL condition without battery.]{\includegraphics[width=0.45\columnwidth]{Figures/1_Week/004.eps}}
\end{figure}
The condition is given by $\Delta t\sum_{i=1}^n\frac{f_i(\bm{x},\bm{u})}{\Delta x_i}\leq1$. In both cases $\Delta t=2^{-11}$.
\end{frame}
}

%------------------------------------------------

{
\setbeamertemplate{footline}{} 
\begin{frame}[noframenumbering]\frametitle{One week states evolution (first week 2019):}
\begin{figure}[ht!]
\centering
\subfloat[Volume of Bonete in optimal path.]{\includegraphics[width=0.45\columnwidth]{Figures/1_Week/106.eps}}\quad
\subfloat[Volume of Palmar in optimal path.]{\includegraphics[width=0.45\columnwidth]{Figures/1_Week/107.eps}}
\end{figure}
\end{frame}
}

%------------------------------------------------

{
\setbeamertemplate{footline}{} 
\begin{frame}[noframenumbering]\frametitle{One week states evolution (first week 2019):}
\begin{figure}[ht!]
\centering
\subfloat[Volume of Salto Grande in optimal path.]{\includegraphics[width=0.45\columnwidth]{Figures/1_Week/108.eps}}\quad
\subfloat[Battery capacity over the optimal path.]{\includegraphics[width=0.45\columnwidth]{Figures/1_Week/122.eps}}
\end{figure}
\end{frame}
}

%------------------------------------------------

{
\setbeamertemplate{footline}{} 
\begin{frame}[noframenumbering]
\frametitle{Approximation by simple functions:}

\begin{columns}[c] % The "c" option specifies centered vertical alignment while the "t" option is used for top vertical alignment

\column{.3\textwidth} % Left column and width
\includegraphics[width=1\textwidth]{Figures/OP_2.eps}\\
\includegraphics[width=1\textwidth]{Figures/OP_4.eps}

\column{.4\textwidth} % Right column and width
We start with large discretizations and we refine when we need.\\
For each refinement we repeat until convergence:
\begin{itemize}

\item Solve the problem forward without fixing the constraint to use theorem (\ref{theo_subgradient}) and compute the subgradient.

\item We supply the subgradient to calculate the next $\lambda$.

\end{itemize}

\end{columns}
\end{frame}
}

%------------------------------------------------

{
\setbeamertemplate{footline}{} 
\begin{frame}[noframenumbering]\frametitle{Opportunity cost (17/01/2019):}
\centering
\includegraphics[width=0.3\columnwidth]{Figures/110.eps}\quad
\includegraphics[width=0.3\columnwidth]{Figures/111.eps}\quad
\includegraphics[width=0.3\columnwidth]{Figures/112.eps}\\
\quad\\
Opportunity cost of the dams, battery and FFSs over the optimal path.
\end{frame}
}

%------------------------------------------------

{
\setbeamertemplate{footline}{} 
\begin{frame}[noframenumbering]\frametitle{Iterations dual problem (17/01/2019):}
\begin{figure}[ht!]
\centering
\subfloat[Accumulated cost using the battery.]{\includegraphics[width=0.45\columnwidth]{Figures/OP_3.eps}}\quad
%\subfloat[Accumulated cost without the battery.]{\includegraphics[width=0.45\columnwidth]{Figures/1_Week/002.eps}}
\end{figure}
\end{frame}
}

%------------------------------------------------

{
\setbeamertemplate{footline}{} 
\begin{frame}[noframenumbering]\frametitle{More ideas from the defense:}
\begin{itemize}

\item  To simulate failure (assume some generator stop working suddenly).

\item Nesterov acceleration for dual problem.

\item Integer constraints represented by non-convex, smooth optimization with many constraints (e.g.: Square and circle intersection in 2 dimensions). Read: $\mathit{l}_p$-Box ADMM: A Versatile Framework for Integer Programming.

\end{itemize}
\end{frame}
}